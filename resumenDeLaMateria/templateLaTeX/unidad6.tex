\saltoPag{}
\section{UNIDAD 6}
    \textbf{\underline{Conceptos:}}
    \begin{itemize}
        \item \textcolor{purple}{\textbf{Solución:}} mezcla homogénea de dos o más sustancias, donde la que está en menor proporción es el soluto y la otra el solvente.
        \item \textcolor{purple}{\textbf{Solución saturada:}} contiene la máxima cantidad de un soluto que se disuelve en un solvente particular y a una temperatura específica.
        \item \textcolor{purple}{\textbf{Solución no saturada:}} contiene menos cantidad de soluto que la que puede disolver el solvente.
        \item \textcolor{purple}{\textbf{Solución sobre-saturada:}} contiene más soluto que el que puede haber en una solución saturada. Son inestables y el exceso de soluto precipita ante un cambio en el sistema.
    \end{itemize}
    \begin{center}
        \begin{tabular}{| m{2cm} | m{2cm} | m{2cm} | m{2cm} |}
            \toprule
            \multicolumn{4}{| c |}{\textcolor{red}{\textbf{\scalebox{0.9}{Clasificación de las soluciones por su estado de agregación}}}} \\
            \midrule
        \multicolumn{1}{|m{1.2cm}|}{} & \multicolumn{1}{ c |}{\textbf{Soluto}} & \multicolumn{1}{ c |}{\textbf{Solvente}} & \multicolumn{1}{ c |}{\textbf{Ejemplos}} \\
            \midrule
            \multicolumn{1}{|m{1.2cm}|}{} & \multicolumn{1}{c|}{Sólido} & \multicolumn{1}{c|}{Sólido} & \multicolumn{1}{c|}{Aleaciones zinc en} \\
            \multicolumn{1}{| c |}{} & \multicolumn{1}{c|}{} & \multicolumn{1}{c|}{} & \multicolumn{1}{c|}{estaño (latón).} \\ \cmidrule(l){2-4}
            \multicolumn{1}{|c|}{\textcolor{blue}{\textbf{Sólidos}}} & \multicolumn{1}{c|}{Gaseoso} & \multicolumn{1}{c|}{Sólido} & \multicolumn{1}{c|}{Hidrógeno en} \\
            \multicolumn{1}{|c|}{} & \multicolumn{1}{c|}{} & \multicolumn{1}{c|}{} & \multicolumn{1}{c|}{paladio.} \\ \cmidrule(l){2-4}
            \multicolumn{1}{|c|}{} & \multicolumn{1}{c|}{Líquido} & \multicolumn{1}{c|}{Sólido} & \multicolumn{1}{c|}{Mercurio en plata} \\
            \multicolumn{1}{|c|}{} & \multicolumn{1}{c|}{} & \multicolumn{1}{c|}{} & \multicolumn{1}{c|}{(amalgama).} \\
            \midrule
            \multicolumn{1}{|c|}{} & \multicolumn{1}{c|}{Líquido} & \multicolumn{1}{c|}{Líquido} & \multicolumn{1}{c|}{Alcohol en agua.} \\ \cmidrule(l){2-4}
            \multicolumn{1}{|c|}{\textcolor{blue}{\textbf{Líquidos}}} & \multicolumn{1}{c|}{Sólido} & \multicolumn{1}{c|}{Líquido} & \multicolumn{1}{c|}{Sal en agua} \\
            \multicolumn{1}{|c|}{} & \multicolumn{1}{c|}{} & \multicolumn{1}{c|}{} &\multicolumn{1}{c|}{(salmuera).} \\ \cmidrule(l){2-4}
            \multicolumn{1}{|c|}{} & \multicolumn{1}{c|}{Gaseoso} & \multicolumn{1}{c|}{Líquido} & \multicolumn{1}{c|}{Oxígeno en agua.} \\
            \midrule
            \multicolumn{1}{|c|}{\textcolor{blue}{\textbf{Gaseosos}}} & \multicolumn{1}{c|}{Gaseoso} & \multicolumn{1}{c|}{Gaseoso} & \multicolumn{1}{c|}{Oxígeno en} \\
            \multicolumn{1}{|c|}{} & \multicolumn{1}{c|}{} & \multicolumn{1}{c|}{} & \multicolumn{1}{c|}{nitrógeno (aire).}  \\
            \bottomrule
        \end{tabular}
    \end{center}
    \subsection{Proceso de disolución}  
        \sangria{} Una solución es un sistema homogéneo constituido por más de un componente. El componente que se encuentra en mayor proporción se denomina \textbf{disolvente} y los que están en menor proporción se llaman \textbf{solutos}. \\
        \sangria{} La facilidad con la que una partícula de \textbf{soluto} sustituye a una molécula de \textbf{disolvente} depende de la fuerza relativa de tres tipos de interacciones:
        \imagen{7cm}{./imagenes/tiposDeInteraccionesProcesoDisolucion.png}
        \imagen{8cm}{./imagenes/procesoDisolucion.png}
        Siendo:
        \ecuacion{\Delta H_{dis} = \Delta H_1 + \Delta H_2 + \Delta H_3} 
        \begin{center}
            $\Delta H_1$ y $\Delta H_2$ = Endotérmico (+) \\
            $\Delta H_3$ = Exotérmico (-)
        \end{center}
        \imagen{8cm}{./imagenes/procesoDisolucionEndoyExotermico.png}
        \midTitle{red}{Proceso de disolución en sólidos}
        \imagen{7cm}{./imagenes/procesoDisolucionSolidos.png}
        \sangria{} El término \textcolor{purple}{''hidratados''} se usa generalmente en iones o moléculas que se encuentran \textcolor{purple}{solvatados} por agua.
        \midTitle{red}{Regla de los semejantes} \vspace{-15pt}
        \midTitle{red}{\textbf{''Lo similar disuelve a lo similar''}}
        \sangria{} Dos sustancias con fuerzas inter-moleculares similares generalmente se disuelven una con la otra.
        \begin{itemize}
            \item Moléculas no polares son solubles en solventes no polares:
                \begin{center} $CCl_4$ en $C_6H_6$ \end{center}
            \item Moléculas polares son solubles en solventes polares.
                \begin{center} $C_2H_5OH$ en $H_2O$ \end{center}
            \item Compuestos iónicos son más solubles en solventes polares.
                \begin{center} $NaCl$ en $H_2O$ o $NH_{3(l)}$ \end{center}
        \end{itemize}
        \saltoPag{}
    \subsection{Solución saturada y solubilidad}
        \sangria{} \textbf{Solubilidad:} cantidad de soluto necesaria para formar una solución saturada en una dada cantidad de disolvente a una temperatura específica.
        \subsubsection{Factores que afectan la solubilidad}
        \begin{itemize}
            \item \textcolor{teal}{\textbf{Interacción soluto-disolvente}}
            \begin{itemize}
                \item \textbf{Gases} como solutos: si no reaccionan con el disolvente la solubilidad aumenta al incrementar la masa molecular. Si ocurre reacción química entre gas y el disolvente, la solubilidad aumenta.
                \item \textbf{Líquidos:} los líquidos polares se disuelven fácilmente en disolventes polares. En los alcoholes al aumentar la longitud de la cadena la solubilidad disminuye; pero si aumenta el número de $OH$, la solubilidad aumenta.
                    Las sustancias no polares son solubles en disolventes no polares.
                \item \textbf{Sólidos:} los sólidos iónicos y polares son solubles en disolventes polares.
            \end{itemize}
            \item \textcolor{teal}{\textbf{Efecto de la presión}} \\
                \sangria{} La solubilidad de un gas en cualquier disolvente aumenta al incrementarse la presión del gas sobre el disolvente.
                \ecuacion{C_g = k P_g}
                Siendo:
                \begin{itemize}
                    \item $C_g$: Concentración del gas $Mol/L$.
                    \item $k$: Constante de proporcionalidad - constante Henry.
                    \item $P_g$: Presión parcial del gas en $atm$.
                \end{itemize}
                \sangria{} La presión no tiene mucho efecto sobre la solubilidades de sólidos y líquidos.
                \imagen{7cm}{./imagenes/efectoPresionSolubilidad.png}
            \item \textcolor{teal}{\textbf{Efecto de la temperatura}} \\
            \sangria{} La solubilidad de la mayoría de los solutos sólidos incrementa al aumentar la temperatura.
            \imagen{7cm}{./imagenes/efectoTemperaturaSolidos.png}
            \sangria{} La solubilidad de los gases en líquidos disminuye con la temperatura (contaminación térmica).
            \imagen{7cm}{./imagenes/solubilidadGasesTemperatura.png}
        \end{itemize}
    \subsection{Unidades de concentración}
        \textcolor{blue}{\textbf{Molaridad (M):}} moles de soluto en $1000mL$ ($1L$) de solución.\\
        \textbf{\underline{Ejemplo:}} \\ Solución de $H_2SO_4$ concentración $0,8M$ $\rightarrow$ Indica que hay $0,8$ moles de $H_2SO_4$ en $1000mL$ de solución. \\[5pt]
        \textcolor{blue}{\textbf{Molalidad (m):}} moles de soluto en $1000g$ ($1Kg$) de solvente. \\
        \textbf{\underline{Ejemplo:}} \\ Solución de $Mg(OH)_2$ concentración $1,5m$ $\rightarrow$ Indica que hay $1,5$ moles de $Mg(OH)_2$ en $1000g$ de solvente (agua). \\[5pt]
        \textcolor{blue}{\textbf{Fracción moles de A:}} 
        \ecuacion{X_A = \frac{n_A}{n_T}} \vspace{5pt}
        \textcolor{blue}{\textbf{Porcentaje peso en peso $\% P/P$:}} gramos de soluto en $100g$ de solución. \\
        \textbf{\underline{Ejemplo:}} \\ Solución de $HNO_3$ concentración $3,4 \% P/P$ $\rightarrow$ Indica que hay $3,4g$ de $HNO_3$ en $100g$ de solución. \\[5pt]
        \saltoPag{}
        \textcolor{blue}{\textbf{Porcentaje peso en volumen \% $P/V$:}} gramos de soluto en 100mL de solución. \\
        \textbf{\underline{Ejemplo:}} \\ Solución de $Na_2CO_3$ concentración 2 \% $P/V$ $\rightarrow$ Indica que hay 2g de $Na_2CO_3$ en $100mL$ de solución. \\[5pt]
        \textcolor{blue}{\textbf{Porcentaje volumen en volumen \% $V/V$:}} $mL$ de soluto en $100mL$ de solución. \\
        \textbf{\underline{Ejemplo:}} \\ Solución de $ClH$ concentración $5\% V/V$ $\rightarrow$ Indica que hay $5mL$ de $ClH$ en $100mL$ de solución.
        \imagen{8cm}{./imagenes/ejemploEjercicioConcentraciones.png}
    \subsection{Dilución}
        \sangria{} Pasar de una solución concentrada a una diluida.
        \imagen{7cm}{./imagenes/dilucion.png}
        \textbf{\underline{Ejemplo:}}
        \imagen{7cm}{./imagenes/ejemploDilusion.png}
        \begin{center}
            \begin{tabular} {| m{8cm} |}
                \toprule
                \multicolumn{1}{| c |}{Los ácidos fuertes en solución, se ionizan liberando} \\
                \multicolumn{1}{| c |}{sus protones ($H^+$)} \\ 
                \multicolumn{1}{|c|}{} \\
                \multicolumn{1}{|c|}{$H_2SO_4 \rightarrow 2{H^+}_{(AC)} + {SO^{-2}}_{4(AC)}$} \\
                \multicolumn{1}{|c|}{$HCl \rightarrow {H^+}_{(AC)} + {Cl^-}_{(AC)}$} \\
                \bottomrule
            \end{tabular}

            \begin{tabular}{|m{8cm}|}
                \toprule
                \multicolumn{1}{|c|}{Las bases fuertes en solución, se ionizan} \\
                \multicolumn{1}{|c|}{liberando sus oxidrilos $(OH^+)$} \\
                \multicolumn{1}{|c|}{} \\
                \multicolumn{1}{|c|}{$Mg(OH)_2 \rightarrow {Mg^{+2}}_{(AC)} + 2(OH)^-_{(AC)}$} \\
                \multicolumn{1}{|c|}{$Na(OH) \rightarrow {Na^+}_{(AC)} + {(OH)^-}_{(AC)}$} \\
                \bottomrule
            \end{tabular}
        \end{center}
        \begin{center}
            \begin{tabular}{|m{8cm}|}
                \toprule
                \multicolumn{1}{|c|}{Las sales en solución se disocian liberando sus} \\
                \multicolumn{1}{|c|}{iones: metálicos y el oxianión o no metal.} \\
                \multicolumn{1}{|c|}{} \\
                \multicolumn{1}{|c|}{$Na_2SO_4 \rightarrow 2{Na^+}_{(AC)} + {SO^{-2}}_{4(AC)}$} \\
                \multicolumn{1}{|c|}{$LiCl \rightarrow {Li^+}_{(AC)} + {Cl^-}_{(AC)}$} \\
                \bottomrule
            \end{tabular}
        \end{center}
        \imagen{7cm}{./imagenes/ejemploDilusion1.png}
    \subsection{Propiedades coligativas}
        \sangria{} Propiedades físicas, las cuales son proporcionales a la concentración (número) de partículas (moléculas o iones) de soluto, e independientemente de su naturaleza.
        \imagen{7cm}{./imagenes/propiedadesColigativas.png}
        \subsubsection{I - Disminución de la presión de vapor}
            \begin{center}
                \begin{tabular}{|m{9cm}|}
                    \toprule
                    \multicolumn{1}{|c|}{La presión de vapor de una solución con un soluto no} \\
                    \multicolumn{1}{|c|}{volátil (no tiene $P_V$ que se pueda medir), es menor} \\
                    \multicolumn{1}{|c|}{que la $P_V$ del solvente puro.} \\
                    \multicolumn{1}{|c|}{} \\
                    \multicolumn{1}{|c|}{\textbf{Ley de Raoult:} La presión parcial del solvente en una} \\
                    \multicolumn{1}{|c|}{solución, es proporcional a su fracción molar.} \\
                    \multicolumn{1}{|c|}{$P_A \propto X_A$} \\
                    \bottomrule
                \end{tabular}
            \end{center}
            \imagen{7cm}{./imagenes/disminucionPresionVapor.png}
            \saltoPag{}
            Siendo:
            \ecuacion{P_A = X_A \cdot {P^o}_{A} \hspace{15pt} X_A = (1 - X_B)}
            \ecuacion{P_A = (1 - X_B) \cdot {P^o}_{A}}
            \ecuacion{P_A = {P^o}_{A} - X_B {P^o}_{A}}
            \ecuacion{{P^o}_{A} - {P}_{A} = X_B{P^o}_{A}}
            \begin{center}
                \begin{tabular}{|m{5cm}|}
                    \toprule
                    \multicolumn{1}{|c|}{$\Delta P_A = X_B {P^o}_{A}$} \\
                    \bottomrule
                \end{tabular}
            \end{center}
            \sangria{} La presión de vapor de un disolvente desciende cuando se le añade un soluto no volátil. Este efecto es el resultado de dos factores:
            \begin{itemize}
                \item La disminución del número de moléculas del disolvente en la superficie libre.
                \item El desorden en la solución es mayor que en el solvente puro lo que disminuye el paso del solvente a la fase vapor.
            \end{itemize}
        \subsubsection{II - Elevación del punto de ebullición (soluto no volátil)}
            \imagen{8cm}{./imagenes/elevacionDelPuntoEbullicion.png}
            \ecuacion{\Delta T_b = T_b - {T^0}_{b}}
            Siendo:
            \begin{itemize}
                \item ${T^0}_{b}$: es el punto de ebullición del solvente puro.
                \item $T_b$: es el punto de ebullición de la solución.
            \end{itemize}
            \ecuacion{Tb > {T^0}_{b} \hspace{15pt} \Delta T_b > 0}
            \begin{center}
                \begin{tabular}{|m{5cm}|}
                    \toprule
                    \multicolumn{1}{|c|}{$\Delta T_b = K_b m$} \\
                    \bottomrule
                \end{tabular}
            \end{center}
            Siendo:
            \begin{itemize}
                \item $m$: es la molalidad de la solución.
                \item $K_b$: es la constante molal de elevación del punto de ebullición, para un solvente dado en ($\text{}^oC/m$)
            \end{itemize}
        \subsubsection{III - Disminución del punto de congelación}
        \imagen{7cm}{./imagenes/disminucionDelPuntoCongelacion.png}
        \ecuacion{\Delta T_f = {T^0}_{f} - T_f}
        Siendo:
        \begin{itemize}
            \item ${T^0}_{f}$: punto de congelación del solvente puro.
            \item $T_f$: punto de congelación de la solución.
        \end{itemize}
        \ecuacion{T_f < {T^0}_{f} \hspace{15pt} \Delta T_f > 0}
        \begin{center}
            \begin{tabular}{|m{5cm}|}
                \toprule
                \multicolumn{1}{|c|}{$\Delta T_f = K_f m$} \\
                \bottomrule
            \end{tabular}
        \end{center}
        Siendo:
        \begin{itemize}
            \item $m$: es la molalidad de la solución.
            \item $K_b$: es la constante molal de elevación del punto de congelación, para un solvente dado en ($\text{}^oC/m$)
        \end{itemize}
        \sangria{} El desorden de las moléculas en la solución es mayor que en el solvente puro. La temperatura de fusión de la solución es menor, porque la cantidad de energía que libera una solución para pasar el líquido a sólido, es mucho mayor que en el solvente puro.
        \imagen{7cm}{./imagenes/algunasPropDeDisolventesComunes.png}
        \saltoPag{}
        \midTitle{red}{Ejercicios}
        \imagen{8cm}{./imagenes/ejerciciosDisolucion1.png}
        \imagen{8cm}{./imagenes/ejerciciosDisolucion2.png}
    \subsection{Presión osmótica ($\pi$)}
        \sangria{} La ósmosis es el paso selectivo de las moléculas del solvente a través de una membrana porosa desde una solución diluida hacia una solución con mayor concentración. \\
        \sangria{} Una membrana semi-permeable permite el paso de las moléculas del solvente, pero se bloque el paso de las moléculas del soluto. \\
        \sangria{} La \textbf{presión osmótica ($\pi$)} es el presión requerida para detener la ósmosis ($atm$).
        \imagen{7cm}{./imagenes/osmosis.png}
        \begin{center}
            \begin{tabular}{|m{5cm}|}
                \toprule
                \multicolumn{1}{|c|}{$\pi = MRT$} \\
                \bottomrule
            \end{tabular}
        \end{center}
        Siendo:
        \begin{itemize}
            \item $M$: concentración molar.
            \item $R$: constante de los gases ideales.
            \item $T$: temperatura.
        \end{itemize}
        \subsubsection{Tipos de disoluciones}
        \imagen{7cm}{./imagenes/globulosRojos.png}
        \textcolor{red}{\textbf{Disoluciones hipotónica}} (menor $\pi$): el agua entra en la célula y se produce la ruptura de las pareces (hemólisis). \\
        \textcolor{red}{\textbf{Disoluciones hipertónica}} (mayor $\pi$): el agua sale de la célula hacia el medio hipertónico (crenación.) La célula puede morir por deshidratación. \\ 
        \textcolor{red}{\textbf{Disoluciones isotónicas}} (igual $\pi$): la solución tienen la misma presión osmótica que la célula. \\
    \subsection{Coloides}
        \sangria{} Los coloides es un estado intermedio entre un solución homogénea y una heterogénea. Puede permanecer en suspensión por largo tiempo estabilizado por interacciones electrostáticas. Un aumento de la concentración iónica puede conducir a la precipitación. \\
        \textcolor{blue}{Un coloide es una dispersión de partículas de una sustancia (Fase dispersa) entre un medio dispersor, formado por otra sustancia. La fase dispersa es demasiado pequeña para ser vista con un microscopio pero lo suficientemente grande para dispersar la luz.} \\
        \textcolor{red}{La fase dispersa y el medio dispersor pueden ser gases, líquidos, sólidos o una combinación de diferentes fases.} \\
        \subsubsection{Tipos de coloides}
            \imagen{8cm}{./imagenes/tiposDeColoides.png}
            \midTitle{blue}{Coloides hidrofílicos}
            \sangria{}Los grupos hidrofílicos de la superficie de una molécula, como la proteína, estabilizan dicha molécula en agua. Estos grupos pueden formar puente hidrógeno con el agua.
            \saltoPag{}
            \imagen{8cm}{./imagenes/coloideHidrofilico.png}
            \midTitle{blue}{Coloide hidrofóbicos}
            \sangria{} En general los coloides hidrofóbicos no son estables en agua y sus partículas forman conglomerados (ej. vinagreta, aceite y agua). Para estabilizarlos, se adsorben iones en la superficie que interactúan con el agua y la repulsión electrostática entre ellos impide que las partículas se junten.
            \imagen{8cm}{./imagenes/coloideHidrofobico.png}
            \midTitle{red}{Acción de limpieza del jabón}
            \sangria{} Los surfactantes como el jabón (estearato sódico), también estabilizan los coloides hidrofóbicos. El cuerpo hidrofóbico es altamente soluble en sustancias aceitosas, las cuales son también no polares, mientras que la cabeza polar permanece fuera de la superficie aceitosa interaccionando con el agua. La mezcla aceite-jabón se estabiliza en el agua formando micelas.
            \imagen{8cm}{./imagenes/limpiezaDelJabon.png}
    \subsubsection{Efecto Tyndall}
        \sangria{} Cuando la luz pasa a través de una disolución verdadera, no se ve dispersión de luz. En una dispersión coloidal la luz es dispersada en muchas direcciones y se observa fácilmente.
        \imagen{6cm}{./imagenes/efectoTyndall.png}


