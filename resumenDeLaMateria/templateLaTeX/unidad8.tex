\saltoPag{}
\section{UNIDAD 8}
    \subsection{Equilibrio químico}
    \sangria{} \textbf{Equilibrio:} es un estado en el cual no se observan cambios a medida que transcurre el tiempo. \\
\sangria{} \textbf{Equilibrio químico:} es un punto que se alcanza cuando: \begin{itemize} \item Los reactivos se transforman en productos con la misma velocidad que los productos vuelven a transformarse en reactivos. \item La concentración de los reactivos y productos permanecen constantes. \end{itemize} \ecuacion{H_2O_{(l)} \rightleftharpoons H_2O_{(g)}} \ecuacion{N_2O_{4(g)} \rightleftharpoons 2NO_{2(g)}} \imagen{7cm}{./imagenes/graficoEquilibrioQuimico.png} \imagen{5cm}{./imagenes/moleculasEquilibrioQuimico.png} \imagen{6cm}{./imagenes/cambioEnLasConcentracionesNO2N2O4.png} \sangria{} Nótese que: \ecuacion{\frac{[NO_2]^2}{[N_2O_4]}} parece permanecer constante a lo largo de la reacción hasta alcanzar el equilibrio. Esto puede llevar a que: \ecuacion{aA + bB \rightleftharpoons cD + dD} \begin{center} \begin{tabular}{|m{6cm}|} \toprule \multicolumn{1}{|c|}{$K = \frac{[C]^c[D]^d}{[A]^a[B]^b}$} \\ \midrule \multicolumn{1}{|c|}{\textbf{Ley de acción de masas}} \\ \bottomrule \end{tabular} \end{center} Si $K >> 1$ significa que favorece la formación de productos. \\ Si $K << 1$ se favorece la formación de reactivos. \imagen{6cm}{./imagenes/favorValoresDeKReactivosProductos.png} \columnbreak{}
    \subsection{Equilibrio homogéneo}
\sangria{} \textbf{Equilibrio homogéneo:} se aplica a las reacciones donde todas las especies reaccionantes se encuentran en la misma fase. \ecuacion{N_2O_{4(g)} \rightleftharpoons 2NO_{2(g)}} \ecuacion{K_c = \frac{[NO_2]^2}{[N_2O_4]} \hspace{15pt} K_p = \frac{P^2_{NO_2}}{P_{N_2O_4}}} En la mayoría de los casos: \ecuacion{K_c \neq K_p} De $P_A V = n_ART$ se obtiene: \ecuacion{P_A = \frac{n_A}{V}RT = [A]RT} \ecuacion{K_p = \frac{[NO_2]^2(RT)^2}{[N_2O_4](RT)}} \ecuacion{K_p = K_c \cdot (RT)^1} \recuadrar{5cm}{$K_p = K_c(RT)^{\Delta n}$} \ecuacion{aA_{(g)} + bB_{(g)} \rightleftharpoons cC_{(g)} + dD_{(g)}} Siendo: \begin{itemize} \item $\Delta n$: moles de productos gaseosos - moles de reactantes gaseosos. \ecuacion{\Delta n = (c + d) - (a + b)} \end{itemize} Si $\Delta n = 0 \Rightarrow K_c = K_p$ \\ \textbf{\underline{Ejemplo:}} \ecuacion{CH_3COOH_{(ac)} + H_2O_{(l)} \rightleftharpoons CH_3COO^-_{(ac)} + H_3O^+_{(ac)}} Entonces: \ecuacion{K_c = \frac{[CH_3COO^-][H_3O^+]}{[CH_3COOH][H_2O]} \hspace{20pt} [H_2O] = \text{ es constante.}} \ecuacion{K_c =\frac{[CH_3COO^-][H_3O^+]}{[CH_3COOH]}} La constante de equilibrio es adimensional. \\ \textbf{\underline{Otro ejemplo:}} \ecuacion{2NO_{(g)} + O_{2(g)} \rightleftharpoons 2NO_{2(g)}} Entonces: \ecuacion{K_c = \frac{[NO_2]^2}{[NO]^2[O_2]} \hspace{15pt} K_p = \frac{P^2_{NO_2}}{P^2_{NO} P_{O_2}} }
    \subsection{Equilibrio heterogéneo}
    \sangria{} \textbf{Equilibrio heterogéneo:} se aplica a las reacciones donde los reactantes y los productos están en diferentes fases. \ecuacion{CaCO_{3(s)} \rightleftharpoons CaO_{(s)} + CO_{2(g)}} Entonces: \ecuacion{K_c = \frac{[CO_2][CaO]}{[CaCO_3]}} \begin{center} \begin{tabular}{|m{8cm}|} \toprule \multicolumn{1}{|m{8cm}|}{La concentración de sólidos y líquidos puros no se considera en la expresión para la constante de equilibrio.} \\ \bottomrule \end{tabular} \end{center} \ecuacion{K_c = [CO_2]} \ecuacion{K_p = P_{CO_2}} \saltoPag{} Siempre que $\Delta n = 0$, $K_c = K_p$.
    \subsection{Equilibrios múltiples}
    \sangria{} Si una reacción puede ser expresada como la suma de dos o más reacciones, la constante de equilibrio de toda la reacción está dada por el producto de las constantes de equilibrio de cada reacción. \begin{center} \begin{tabular}{c c} \multicolumn{1}{c}{$A + B \rightleftharpoons \cancel{C} + \cancel{D}$} & \multicolumn{1}{c}{$K^{\prime}_c$} \\ \multicolumn{1}{c}{$\cancel{C} + \cancel{D} \rightleftharpoons E + F$} & \multicolumn{1}{c}{$K^{\prime\prime}_c$} \\ \midrule \multicolumn{1}{c}{$A + B \rightleftharpoons E + F$} & \multicolumn{1}{c}{$K_c$} \\ \end{tabular} \end{center} \begin{center} \begin{tabular}{c c c} \multicolumn{1}{c}{$K_c^{\prime} = \frac{[C][D]}{[A][B]}$} & \multicolumn{1}{c}{$K_c^{\prime\prime} = \frac {[E][F]}{[C][D]}$} & \multicolumn{1}{c}{$K_c = \frac{[E][F]}{[A][B]}$} \\ \end{tabular} \end{center}
    \midTitle{red}{Representación de $K$ y la ecuación de equilibrio}
    \ecuacion{N_2O_{4(g)} \rightleftharpoons 2NO_{2(g)} \hspace{15pt} 2NO_{2(g)} \rightleftharpoons N_2O_{4(g)}} \ecuacion{K_c = \frac{[NO_2]^2}{[N_2O_4]} = 4,63 \times 10^{-3}}\ecuacion{K^{\prime}_c = \frac{[N_2O_4]}{[NO_2]^2} = \inv{K_c} = 216} \sangria{} Cuando la ecuación de una reacción reversible está escrita en dirección opuesta, la constante de equilibrio se convierte en el recíproco de la constante de equilibrio original. \ecuacion{N_2O_{4(g)} \rightleftharpoons 2NO_{2(g)} \hspace{15pt} K_c = \frac{[NO_2]^2}{[N_2O_4]} = 4,63 \times 10^{-3}} \ecuacion{\unMedio{} N_2O_{4(g)} \rightleftharpoons NO_{2(g)} \hspace{15pt} K^{\prime}_c = \frac{[NO_2]}{[N_2O_4]^{\unMedio{}}} = 0,0680} \ecuacion{K^{\prime}_c = (K_c)^{\unMedio{}}} \sangria{} El valor de $K$ depende de cómo esté balanceada la ecuación del equilibrio.
    \subsection{Expresiones de constante de equilibrio}
        \begin{enumerate} \item Las concentraciones de las especies en reacción en la fase condensada se expresan en $M$. En la fase gaseosa, las concentraciones pueden ser expresadas con $M$ o en $atm$. \item Las concentraciones de los sólidos puros, líquidos puros y solventes no aparecen en las expresiones de constante de equilibrio. \item La constante de equilibrio es una cantidad sin dimensiones. \item Al calcular el valor de la constante de equilibrio, se debe especificar la ecuación balanceada y la temperatura. \item Si una reacción puede ser expresada como la suma de dos o más reacciones, la constante de equilibrio para la reacción global está determinada por el producto de las constantes de equilibrio de cada una de las reacciones. \end{enumerate} 
    \subsection{Cociente de reacción ($Q$)}
\sangria{} El cociente de una reacción ($Q_c$) se calcula sustituyendo las concentraciones iniciales de los reactantes y productos en la expresión de la constante de equilibrio ($K_c$). \begin{itemize} \item \textcolor{purple}{\textbf{Si $Q_c = K_c$:}} el sistema se encuentra en equilibrio. \item \textcolor{purple}{\textbf{Si $Q_c < K_c$:}} el sistema evolucionará hacia la derecha, es decir, aumentarán las concentraciones de los productos y disminuirán las de los reactivos hasta que $Q$ se iguale con $K_c$. \item \textcolor{purple}{\textbf{Si $Q_c > K_c$:}} el sistema evolucionará hacia la izquierda, es decir, aumentarán las concentraciones de los reactivos y disminuirán las de los productos hasta que $Q$ se iguale con $K_c$. \end{itemize} \imagen{6cm}{./imagenes/cocienteDeReaccion.png} \textbf{\underline{Ejemplo:}} \\ La constante de equilibrio ($K_c$) para la reacción: \ecuacion{2NO_{(g)} + Cl_{2(g)} \rightleftharpoons 2NOCl_{(g)}} es de $6,5 \times 10^{-4}$ a $35^oC$. Si en un recipiente de $2L$ se mezclan $2 \times 10^{-2}$ moles de $NO$, $8,3 \times 10^{-3}$ moles de $Cl_2$ y $6,8$ moles de $NOCl$ (Cloruro de nitrosilo) ¿El sistema está en equilibrio? Si no es así, ¿En qué dirección el sistema alcanzará el equilibrio?\ecuacion{[NO]_0 = \frac{2 \times 10^{-2} moles}{2L} = 10^{-2}M \hspace{15pt} Q_c = \frac{[NOCl]_0^2}{[NO]_0^2[Cl_2]_0}} \ecuacion{[Cl_2]_0 = \frac{8,3 \times 10^{-3}moles}{2L} = 4,15 \times 10^{-3}M \hspace{15pt} Q_c = \frac{[3,4 \times 10^{-3}]^2}{[10^{-2}]^2[4,15 \times 10^{-3}]}} \ecuacion{[NOCl]_0 = \frac{6,8 \times 10^{-3}moles}{2L} = 3,4 \times 10^{-3}M \hspace{15pt} Q_c = 27,85 > K_c} \sangria{} El sistema evolucionará de derecha a izquierda hacia los reactivos, disminuyendo la cantidad de productos hasta alcanzar el equilibrio.
    \subsection{Cálculo de concentraciones de equilibrio} 
        \begin{enumerate}
            \item Expresar las concentraciones de equilibrio de todas las especies en términos de las concentraciones iniciales y como una incógnita $x$, que representa el cambio de concentración.
            \item Escribir la expresión de la constante de equilibrio en términos de las concentraciones de equilibrio. Sabiendo el valor de la constante de equilibrio, despejar $x$.
            \item Teniendo el valor de $x$, calcular las concentraciones de equilibrio de todas las especies.
        \end{enumerate} \saltoPag{}
        \textbf{\underline{Ejercicio:}} A $1280^oC$ la constante de equilibrio ($K_c$) para la reacción: \ecuacion{Br_{2(g)} \rightleftharpoons 2Br_{(g)}} es de $1,1 \times 10^{-3}$. Si las concentraciones iniciales son $[Br_2] = 0,063M$ y $[Br] = 0,012M$, calcular las concentraciones de estas especies en equilibrio.
        \ecuacion{Q_c = \frac{[0,012]_0^2}{[0,063]_0}} \ecuacion{Q_c = 2,2 \times 10^{-3}} \begin{center} $Q_c > K_c$ \\[3pt] El sistema se desplaza hacia los reactivos. \end{center} Despejamos la ''$x$'' como el cambio en la concentración de $Br_2$:
        \begin{center}
            \begin{tabular}{c c c}
                \multicolumn{1}{c}{} & \multicolumn{2}{c}{$Br_{2(g)} \rightleftharpoons 2Br_{(g)}$} \\[5pt] \multicolumn{1}{c}{Inicial($M$)} & \multicolumn{1}{c}{0,063} & \multicolumn{1}{c}{0,012} \\ \multicolumn{1}{c}{Cambio($M$)} & \multicolumn{1}{c}{$x$} & \multicolumn{1}{c}{$-2x$} \\ \midrule \multicolumn{1}{c}{Equilibrio($M$)} & \multicolumn{1}{c}{$0,063 + x$} & \multicolumn{1}{c}{$0,012 - 2x$} \\
            \end{tabular}
        \end{center}
        \ecuacion{K_c = \frac{(0,012 - 2x)^2}{0,063 + x} = 1,1 \times 10^{-3}}
        \ecuacion{4x^2 - 0,0491x `0,000144 = 0,0000693 + 0,0011x}
        \ecuacion{4x^2 - 0,0491x + 0,0000747 = 0}
        \ecuacion{x = \frac{-b \pm \sqrt{b^2 - 4ac}}{2a}}
        \ecuacion{\cancel{x_1 = 0,0105} \hspace{15pt} x_2 = 0,00178}
        En equilibrio, $[Br] = 0,012 - 2x = \cancel{-0,009M}$ Concentración negativa. \\ En equilibrio, $[Br_ 2] = 0,064 + x = 0,0735M$ 
        Respuestas de concentraciones: \recuadrar{4cm}{$0,00844M$} \recuadrar{4cm}{$0,0648M$}

    \subsection{Principio de Le Chatelier}
        \sangria{} \textit{''Si un sistema en equilibrio se somete a un cambio de temperatura, presión o concentración de uno de los componentes, el sistema desplazará su posición de equilibrio de modo que se contrarreste el efecto de la modificación.''} 
        \subsubsection{Cambio en la concentración}
        \midTitle{red}{\textbf{El valor de $K_c$ no varía}} \sangria{} Si una vez establecido un equilibrio se varía la concentración de algún reactivo o producto, el equilibrio desaparece, la reacción evoluciona hacía el nuevo equilibrio.
        \textbf{\underline{Ejemplo:}} \ecuacion{N_{2(g)} + 3H_{2(g)} \rightleftharpoons 2NH_{3(g)} \hspace{20pt} K_c = 2,37 \times 10^{-3}} \ecuacion{[N_2]_{eq} = 0,683M \hspace{10pt} [H_2]_{eq} = 8,8M \hspace{10pt} [NH_3]_{eq} = 1,05}
        \ecuacion{Q_c = \frac{[NH_3]^2}{[N_2][H_2]^3}}
        \ecuacion{Q_c = \frac{[3,65]^2}{[0,683][8,8]^3}}
        \ecuacion{Q_c = 2,86 \times 10^{-2} > K_c= 2,37 \times 10^{-3}}
        Se consume $NH_3$ para generar $N_2$ e $H_2$, la reacción se desplaza hacia la izquierda.
        \subsubsection{Cambio en la presión (o volumen)}
        \midTitle{blue}{\textbf{El valor de $K_c$ no varía}} \sangria{} En un equilibrio químico en el cual el número de moles gaseosos entre reactivos y productos no son iguales: \ecuacion{Br_{2(g)} \rightleftharpoons 2Br_{(g)}} \ecuacion{P = \frac{[n]RT}{[V]} \hspace{15pt} Q_c = \frac{[Br]^2}{[Br_2]}} \sangria{} Al aumentar la presión o disminuir el volumen, la concentración aumenta y en el caso el numerador incrementa en mayor proporción que denominador $Q>K$ el sistema se desplaza hacia los reactivos. Se favorece el menor $n^o$ de moles gaseosos. \\ \sangria{} Al disminuir la presión o aumentar el volumen, $Q<K$ el sistema se desplaza hacia los productos. Se favorece el mayor $n^o$ de moles gaseosos. \\ \sangria{} Si el número de moles gaseosos de reactivos es igual al de productos, un cambio de presión o volumen no afecta al equilibrio ni la expresión de $K_c$.
        \subsubsection{Cambio en la temperatura} \midTitle{purple}{\textbf{Modifica el valor de $K_c$}}
        \textbf{\underline{Reacción endotérmica}} \ecuacion{\textcolor{red}{Calor} + N_2O_{4(g)} \rightarrow 2NO_{2(g)} \hspace{15pt} \Delta H = 58kJ} \ecuacion{K_c = \frac{[NO_2]^2}{[N_2O_4]}} \begin{center} Si $\Delta H > 0$ (endotérmico): $\uparrow T$ $\uparrow K_c$ \hspace{10pt} $\downarrow T$ $\downarrow K_c$  \end{center}
        \sangria{} En una reacción endotérmica, al aumentar la $T$ el sistema se desplaza hacia los productos y $K_c$ aumenta, se busca consumir el calor agregado en los reactivos. Si se disminuye $T$ el sistema se desplaza hacia los reactivos y $K_c$ disminuye. \\[5pt]
        \textbf{\underline{Reacción exotérmica}}
        \ecuacion{H_{2(g)} + Cl_2 \rightarrow 2HCl_{(g)} + \textcolor{red}{Calor} \hspace{15pt} \Delta H = -185kJ}  \ecuacion{K_c = \frac{[HCl]^2}{[H_2][Cl_2]}} \begin{center} Si $\Delta H < 0$ (exotérmico): $\uparrow T$ $\downarrow K_c$ \hspace{10pt} $\downarrow T$ $\uparrow K_c$ \end{center} \sangria{} En una reacción exotérmica al aumentar la $T$ el sistema se desplaza hacia los reactivos y $K_c$ disminuye. Se busca disminuir el aumento de calor en los productos. Si se disminuye la $T$ el sistema se desplaza hacia los productos y $K_c$ aumenta. \saltoPag{}
        \subsubsection{Agregado de un gas inerte}
            \begin{center}
                \begin{tabular}{cc}
                    \multicolumn{1}{c}{$P_t = P_A + P_B + P_{inerte}$} & \multicolumn{1}{m{4cm}}{La presión total aumenta.} \\[10pt]
                    \multicolumn{1}{c}{$X_A = \frac{n_A}{(n_A + n_B + m_{inerte})}$} & \multicolumn{1}{m{4cm}}{Las fracciones molares disminuyen.} \\[10pt]
                    \multicolumn{1}{c}{$P_A = X_A P_T$} & \multicolumn{1}{m{4cm}}{Las presiones parciales no cambian $\rightarrow$ $K_p$ no cambia.} \\[10pt]
                    \multicolumn{1}{c}{$C_A = \frac{n_A}{V}$} & \multicolumn{1}{m{4cm}}{Las concentraciones no cambian $\rightarrow$ $K_c$ no cambia.} \\
                \end{tabular}
            \end{center}
            \sangria{} Para un sistema de volumen fijo y en estado de equilibrio, el agregado de un gas inerte produce el aumento de la presión total del sistema y la disminución de las fracciones molares de los gases. La concentración no cambia y las presiones parciales tampoco. Por lo tanto, no modifica el equilibrio.
            \subsubsection{Agregado de un catalizador}
                \sangria{} Un catalizador aumenta la velocidad de reacción disminuyendo la energía de activación. Se aumenta la velocidad directa e inversa. No modifica la $K_c$ ni desplaza la posición de equilibrio.
                \midTitle{black}{Importancia en procesos industriales}
                \sangria{} Es muy importante en la industria el saber qué condiciones favorecen el desplazamiento de un equilibrio hacia la formación de un producto, pues se conseguirá un mayor rendimiento, en dicho proceso. \\
                \begin{itemize} \item Síntesis de Haber en la formación de amoniaco. \end{itemize} \ecuacion{N_{2(g)} + 3H_{2(g)} \rightleftharpoons 2NH_{3(g)} \hspace{15pt} \Delta H < 0} \sangria{} La formación de amoniaco está favorecida por altas presiones y por una baja temperatura. Por ello esta reacción se lleva a cabo a altísima presión y a una temperatura relativamente baja, aunque no puede ser muy baja para que la reacción no sea muy lenta. Hay que mantener un equilibrio entre rendimiento y tiempo de reacción.

    \subsection{Ácidos y bases}
    \sangria{} Un \textcolor{red}{Ácido} de Brønsted es una sustancia que puede donar un protón. \\ \sangria{} Una \textcolor{blue}{base} de Brønsted es una sustancia que puede aceptar un protón. \imagen{6cm}{./imagenes/acidosYBases.png}
    \subsubsection{Propiedades ácido-base del agua}
        \ecuacion{H_2O_{(l)} \rightleftharpoons H^+_{(ac)} + OH^-_{(ac)}}
        \midTitle{blue}{Auto-ionización del agua}
        \imagen{8cm}{./imagenes/autoionizacionDelAgua.png}
        \midTitle{red}{Producto iónico del agua}
        \ecuacion{H_2O_{(l)} \rightleftharpoons H^+_{(ac)} + OH^-_{(ac)}}
        \ecuacion{K_c = \frac{[H^+][OH^-]}{[H_2O]} \hspace{15pt} [H_2O] = \text{constante}}
        \ecuacion{K_c[H_2O] = K_w = [H^+][OH^-]}
        \sangria{} La constante del producto iónico ($K_w$) es el producto de la concentración molar de los iones $H^+$ y $OH^-$ a una temperatura en particular. En el agua pura a $25^oC$, la $[H^+] = 10^{-7}M$ y $[OH^-]= 10^{-7}M$
        \ecuacion{K_w = [10^{-7}][1o^{-7}] = 10^{-15}}
        \sangria{} Independientemente de que se trate de agua pura o de una solución acuosa, la siguiente expresión siempre se cumple a $25^oC$: \ecuacion{K_w = [H^+][OH^-] = 10^{-14}}
    \subsubsection{El $pH$: medida de la acidez}
    \sangria{} Debido a que las concentraciones de $[H^+]$ y $[OH^-]$ son número muy pequeños, se propuso el $pH$ como medida más práctica. El $pH$ es el logaritmo negativo de la concentración de $[H^+]$ en $Mol / L$. El signo negativo da un valor positivo de $pH$.
    \recuadrar{5cm}{$K_w = 10^{-14} = [H^+][OH^-]$} \begin{center} Aplicando la función $p = -log$ \end{center}
    \ecuacion{pK_w = p[H^+] + p[OH^-]}
    \ecuacion{-log[10^{-14}] = pH + pOH}
    \recuadrar{4cm}{$14 = pH + pOH$}
    \imagen{5cm}{./imagenes/phYPOH.png} \saltoPag{}
    \sangria{} Si la concentración de $[H^+]$ aumenta (disociación de un ácido), entonces la concentración de $[OH^-]$ debe disminuir para que el producto de ambas concentraciones continúe siendo $10^{-14}$.
    \midTitle{blue}{Variación de las concentraciones de $H^+$ y $OH^-$ a $25^oC$}
    \sangria{} Las concentraciones de iones hidrógeno $[H^+]$ e hidróxidos $[OH^-]$ cambian manteniendo el valor de $K_w$ constante a una temperatura dada.
    \imagen{8cm}{./imagenes/variacionPHyPOH.png}
    \subsection{Electrolitos fuertes y débiles}
    \begin{itemize}
        \item \textcolor{red}{\textbf{Electrolitos fuertes:}} se ionizan completamente en solución acuosa ($\rightarrow$):
            \ecuacion{HCl_{(ac)} \rightarrow H^+_{(ac)} + Cl^-_{(ac)}}
            \ecuacion{HNO_{3(ac)} \rightarrow H^+_{(ac)} + NO^-_{3(ac)}}
            \ecuacion{H_2SO_{4(ac)} \rightarrow 2H^+_{(ac)} + SO^{2-}_{4(ac)}}
            \ecuacion{NaOH_{(ac)} \rightarrow Na^+ + OH^-}
        \item \textcolor{red}{\textbf{Electrolitos débiles:}} se ionizan en forma limitada en solución acuosa ($\longleftrightarrow$). Están disociados parcialmente:
            \ecuacion{CH_3-COOH_{(ac)} \longleftrightarrow CH_-COO^- + H^+}
            \ecuacion{NH_{3(ac)} + H_2O \longleftrightarrow NH_4 + OH^-}
    \end{itemize}
    \midTitle{red}{Cálculo de $pH$ en electrolitos fuertes}
    \sangria{} Calcular el $pH$ de una solución de $Ba(OH)_2$ $0,02M$:
    \begin{center}
        \begin{tabular}{cccc}
            \multicolumn{1}{c}{} & \multicolumn{3}{c}{$Ba(OH)_{2(s)} \longrightarrow Ba^{2+}_{(ac)} + 2(OH)^-_{(ac)}$} \\ 
            \multicolumn{1}{c}{\scalebox{0.8}{Conc. inic. (M)}} & \multicolumn{1}{c}{$0,02$} & \multicolumn{1}{c}{$0$} & \multicolumn{1}{c}{$0$} \\
            \multicolumn{1}{c}{\scalebox{0.8}{Cambio (M)}} & \multicolumn{1}{c}{$-0,02$} & \multicolumn{1}{c}{$0,02$} & \multicolumn{1}{c}{$2(0,02)$} \\ \midrule
        \multicolumn{1}{c}{\scalebox{0.8}{Conc. Final (M)}} & \multicolumn{1}{c}{$0$} & \multicolumn{1}{c}{$0,02$} & \multicolumn{1}{c}{$0,04$} \\
        \end{tabular}
    \end{center}
    \ecuacion{[OH^-] = 0,04M}
    \ecuacion{pOH = -log[OH^-]}
    \recuadrar{3cm}{$pOH = 1,4$}
    \ecuacion{pH = 14 - pOH}
    \ecuacion{pH = 14 - 1,4}
    \columnbreak{}
    \recuadrar{3cm}{$pH = 12,6$} \begin{center} $pH >$  7 $\Rightarrow$ Solución básica \end{center}

    \subsection{Equilibrio de solubilidad}
    \sangria{} El equilibrio se da entre un sólido poco soluble y sus iones en solución.
    \begin{itemize}
        \item Son reacciones de equilibrio heterogéneo sólido-líquido.
        \item La fase sólido contiene una sustancia poco soluble (normalmente una sal).
        \item La fase líquida contiene los iones producidos en la disociación de la sustancia sólida.
        \item Normalmente el disolvente es agua.
    \end{itemize}
    \subsubsection{Producto de solubilidad ($K_{ps}$)}
    \ecuacion{AgCl_{(s)} \rightleftharpoons Ag^+_{(ac)} + Cl^-_{(ac)}}
    \recuadrar{4cm}{$K_{ps} = [Ag^+][Cl^-]$}
    \begin{center} $K_{ps}$ es la \textbf{constante del producto de solubilidad} \end{center}
    Siendo:
    \begin{itemize}
        \item $[Ag^+]$: concentración molar de $Ag^+$ en la solución.
        \item $[Cl^-]$: concentración molar de $Cl^-$ en la solución.
    \end{itemize}
    \ecuacion{MgF_{2(s)} \rightleftharpoons Mg^{2+}_{(ac)} + \textcolor{red}{2}F^-_{(ac)}}
    \ecuacion{K_{ps} = [Mg^{2+}][F^-]^{\textcolor{red}{2}}}
    \imagen{8cm}{./imagenes/tablaKps.png}
    \begin{center} Algunos ejemplos de $K_{ps}$ \end{center}
    \midTitle{purple}{Disolución de un sólido iónico en una solución acuosa}
    \sangria{} $Q =$ producto iónico, tiene la misma expresión que $K_{ps}$ con concentraciones que no son las de equilibrio.
    \begin{center}
        \begin{tabular}{ccc}
            \multicolumn{1}{c}{$Q<K_{ps}$} &
            \multicolumn{1}{c}{Solución no saturada} &
            \multicolumn{1}{c}{No hay precipitado} \\[5pt]

            \multicolumn{1}{c}{$Q = K_{ps}$} &
            \multicolumn{1}{c}{Solución saturada} &
            \multicolumn{1}{c}{} \\[5pt]

            \multicolumn{1}{c}{$Q>K_{ps}$} &
            \multicolumn{1}{c}{Solución sobresaturada} &
            \multicolumn{1}{c}{Hay precipitado} \\[5pt]

        \end{tabular}
    \end{center}
    \saltoPag{}
    \midTitle{red}{Solubilidad ($S$)}
    \textbf{Solubilidad molar ($mol/L$)}: es el número de moles de soluto disueltos en $1L$ de una solución saturada. \\
    \textbf{Solubilidad ($g/L$)}: es el número de gramos de soluto disueltos en $1L$ de solución saturada.
    \midTitle{red}{Producto de solubilidad ($K_{ps}$) en electrolitos de tipo $AB$}
    \sangria{} En un electrolito de tipo $AB$ el equilibrio de solubilidad viene determinado por: \\
    \textbf{\underline{Ejemplo:}}
    \begin{center}
        \begin{tabular}{cccc}
            \multicolumn{1}{c}{} &
            \multicolumn{3}{c}{$AgCl_{(s)} \rightleftharpoons Ag^+_{(ac)} + Cl^-_{(ac)}$} \\

            \multicolumn{1}{c}{Conc. inic. ($mol/L$):} &
            \multicolumn{1}{c}{} &
            \multicolumn{1}{c}{$0$} &
            \multicolumn{1}{c}{$0$} \\

            \multicolumn{1}{c}{Cambio:} &
            \multicolumn{1}{c}{$-s$} &
            \multicolumn{1}{c}{$+s$} &
            \multicolumn{1}{c}{$+s$} \\ \midrule

            \multicolumn{1}{c}{Conc. eq. ($mol/L$):} &
            \multicolumn{1}{c}{} &
            \multicolumn{1}{c}{$+s$} &
            \multicolumn{1}{c}{$+s$} \\
            
        \end{tabular}
    \end{center}
    La concentración del sólido permanece constante.
    \ecuacion{K_{ps} = [Ag^+][Cl^-]}
    \ecuacion{K_{ps} = [s][s]}
    \ecuacion{K_{ps} = [s]^2}
    \recuadrar{3cm}{$s = \sqrt{K_{ps}}$}
    \midTitle{purple}{Predicción de reacciones de precipitación}
    \textbf{\underline{Ejemplo:}} \\
    Deducir si se formará precipitado de $AgCl$ $(K_{ps} = 1,7 \times 10^{-10})$ a $25^oC$ al añadir a $250cm^3$ de $NaCl 0,02M$ a $50cm^3$ de $AgNO_3 0,5M$
    \ecuacion{AgCl_{(s)} \rightleftharpoons Ag^+_{(ac)} + Cl^-_{(ac)}}
    \ecuacion{K_{ps} = [Ag^+][Cl^-] = s^2}
    \ecuacion{n(Cl^-) = 0,25L \cdot 0,02 mol/L = 0,005 mol}
    \ecuacion{[Cl^-] = \frac{0,005mol}{0,25L + 0,05L} = 0,0167M}
    Igualmente: $n(Ag^+) = 0,05L \cdot 0,5 mol/L = 0,025 mol$
    \ecuacion{[Ag^+] = \frac{0,025mol}{0,25L + 0,05L} = 0,0833M}
    \ecuacion{Q_{ps} = [Ag^+][Cl^-] = (0,0167)(0,0833) = 1,39 \times 10^{-3}}
    Como $Q_{ps} > K_{ps}$ entonces precipitará. \\[5pt]
    \textbf{\underline{Otro ejemplo:}} \\[5pt]
    \textcolor{blue}{¿Si $0,05L$ de $NaOH 0,03M$ son agregados a $1L$ de $CaCl_2 0,01M$, se formará un precipitado?} \\
    Los iones presentes en la solución son: $Na^+, OH^-, Ca^{2+}, Cl^-$ \\
    El único precipitado posible es $Ca(OH)_2$ ya que el $NaCl$ es muy soluble.
    \ecuacion{Ca(OH)_2 \rightleftharpoons Ca^{2+} + 2(OH)^-}
    Cálculo de las concentraciones de los iones en el volumen final de mezcla $1,05L$
    \columnbreak{}
    \ecuacion{[OH^-]_0 = (0,03\frac{mol}{L} \times 0,05L)/1,05L = 1,43 \times 10^{-3}M}
    \ecuacion{[Ca^{2+}]_0 = (0,01\frac{mol}{L} \times 1L)/1,05L = 9,52 \times 10^{-3}M}
    \ecuacion{Q = [Ca^2+]_0[OH^-]^2_0 = (9,52 \times 10^{-3}) \times (1,43 \times 10^{-3}) = 1,9 \times 10^{-8}}
    \ecuacion{K_{ps} = [Ca^{2+}][OH^-]^2 = 8,0 \times 10^{-6} \text{ Dato de tabla}}
    $Q<K_{ps}$ No se forma precipitado.
    
