\saltoPag{}
\section{UNIDAD 5}
    \subsection{Conceptos}
        \begin{itemize}
            \item \textbf{Energía:} capacidad para efectuar trabajo.
            \item \textbf{Energía térmica:} es la energía asociada con el movimiento arbitrario de átomos y moléculas.
            \item \textbf{Energía potencial:} es la energía disponible en virtud de la posición de un objeto.
            \item \textbf{Energía solar:} energía radiante proveniente del sol y es una fuente de energía primaria de la Tierra.
            \item \textbf{Energía química:} energía almacenada dentro de los enlaces de las sustancias químicas.
            \item \textbf{Termoquímica:} estudia los cambios de energía(calor) de las reacciones químicas.
        \end{itemize}
        \midTitle{blue}{Cambios de energía en las reacciones químicas}
        \begin{itemize}
            \item \textbf{Calor:} transferencia de energía térmica entre dos cuerpos que se encuentran a temperaturas diferentes.
            \item \textbf{Sistema:} parte pequeña del universo que se aísla para someter a estudio. Pueden ser:
                \begin{itemize}
                    \item Abiertos: intercambio de masa y energía.
                    \item Cerrados: intercambio de energía.
                    \item Aislados: no hay intercambio de masa ni de energía.
                \end{itemize}
                \imagen{7cm}{./imagenes/tiposDeSistemas.png}
            \item \textbf{Entorno o alrededores:} resto del universo externo al sistema.
        \end{itemize}
        \textbf{\underline{Ejemplo:}} la combustión del Hidrógeno gaseoso con oxígeno, libera gran cantidad de energía.
        \begin{center} 
            $2H_{2(g)} + O_{2(g)} \rightarrow 2H_2O_{l} +$ ENERGÍA. 
        \end{center}
        \sangria{} La mezcla de reacción es un sistema y el resto del universo, los alrededores. El calor generado por el sistema lo toma los alrededores. El proceso es \textbf{Exotérmico}.
        \sangria{} La descomposición de óxido de Mercurio, es un proceso endotérmico.
        \imagen{7cm}{./imagenes/combustionHidrogenoOxigenoYDescomposicionOxidoMercurio.png}
    \subsection{Termodinámica}
        \sangria{} La termodinámica es el estudio de la conversión del calor y otras formas de energía. Examina los cambios en el \textbf{estado de un sistema}.
        \subsubsection{Primera ley de la termodinámica}
            \sangria{} La energía se puede convertir de una forma a otra, pero no se puede crear ni destruir, solo se conserva. Cualquier energía que un sistema pierda deberá ser ganada por el entorno y viceversa. \\
            \textbf{Energía interna:} es la suma de la energía cinética (movimiento molecular y de los electrones) y potencial (fuerzas de atracción entre electrones y núcleos, fuerzas de repulsión entre electrones y entre núcleos de moléculas e interacciones entre moléculas).
            La energía no se mide de forma absoluta, se miden los cambios de energía.
            \ecuacion{S_{(s)} + O_{2(g)} \rightarrow SO_{2(g)} \text{Reacción exotérmica}}
            \ecuacion{\Delta E = E_{(productos)} - E_{(reactivos)}}
            \ecuacion{\Delta E_{(sis)} + \Delta E_{(alr)} = 0}
            \midTitle{blue}{Cambio de energía de un sistema}
            \ecuacion{\Delta E = q + W}
            \sangria{} El cambio de $E$ interna de un sistema lo podemos expresar como la suma del intercambio de calor ($q$) y el trabajo ($w$) realizado por o sobre el sistema.
            \imagen{6cm}{./imagenes/sistema.png}
            \begin{itemize}
                \item \textbf{Trabajo:} cambio de energía producido por un proceso.
                \begin{itemize}
                    \item Signo del Trabajo:
                        \begin{itemize}
                            \item Compresión ($\Delta V < 0$), signo ($-$) $\Rightarrow W (+)$
                            \item Expansión ($\Delta V > 0$), signo ($+$) $\Rightarrow W(-)$
                        \end{itemize}
                \end{itemize}
            \end{itemize}
            \imagen{7cm}{./imagenes/deltaVolumenTrabajoGas.png}
            \saltoPag{}
            \midTitle{blue}{Calor a volumen constante ($q_v$)}
            \begin{center} $\Delta E = q + W$ \\[5pt] \end{center}
            Siendo:
            \begin{center} $W = - P \Delta V$ \\[5pt] $W = 0$ \end{center}
            Entonces:
            \begin{center}
                \begin{tabular}{| c |}
                    \hline 
                    $\Delta E = q_V$ \\ 
                    \hline 
                \end{tabular} 
            \end{center}
            \midTitle{blue}{Calor a presión constante ($q_P$): Entalpia}
            \begin{center} $\Delta E = q + W$ \end{center}
            Siendo:
            \begin{center}
                $q_P = \Delta H$ \\[5pt]
                $W = -P \Delta V$
            \end{center}
            Entonces:
            \begin{center}
                $\Delta E = \Delta H - P \Delta V$ \\[5pt]
                \begin{tabular}{| c |}
                    \hline
                    $\Delta H = \Delta E + P \Delta V $ \\
                    \hline
                \end{tabular}
            \end{center}
            Para sistemas gaseosos:
            \begin{center} \begin{tabular}{| c |} \hline $\Delta H = \Delta E + R \cdot T \cdot \Delta n$ \\ \hline \end{tabular} \end{center}
            \midTitle{blue}{Entalpia ($H$) = $q_P$}
            \sangria{} Es la cantidad de calor medido en condiciones de presión estándar.
            \ecuacion{\Delta H = H_{(productos)} - H_{(reactivos)}}
            Siendo:
            \begin{itemize} 
                \item $\Delta H$: calor liberado o absorbido a presión constante. 
            \end{itemize}
            \imagen{7cm}{./imagenes/entalpiaEjemplo.png}
        
    \subsection{Ecuaciones termoquímicas}
        \sangria{} Cada mol de hielo que se funde a $0^oC$ y $1atm$ absorbe $6,01kJ$
        \ecuacion{H_2O_{(s)} \rightarrow H_2O_{(l)} \hspace{15pt} \Delta H = 6,01 kJ / mol}
        \begin{center} El sistema absorbe calor, es endotérmico $\Delta H > 0$ \end{center}
        \sangria{} Para cada mol de metano que se quema a $25^oC$ y $1atm$ se liberan $-890,4 kJ$.
        \ecuacion{CH_{4(g)} + 2O_{2(g)} \rightarrow CO_{2(g)} + 2H_2O_{(l)} \hspace{15pt} \Delta H = -890,4 kJ / mol}
        \begin{center} El sistema emite calor, es exotérmico $\Delta H < 0$ \end{center}
        \begin{itemize}
            \item Los coeficientes estequiométricos siempre se refieren al número de moles de una sustancia.
                \begin{center} $H_2O_{(s)} \rightarrow H_2O_{(l)} \hspace{15pt} \Delta H = 6,01 kJ$ \end{center}
            \item Si se invierte una reacción, el signo de $\Delta H$ también se invierte.
                \ecuacion{H_2O_{(l)} \rightarrow{} H_2O_{(s)} \hspace{15pt} \Delta H = -6,01 kJ}
            \item Si se multiplican ambos lados de la ecuación por un factor, entonces $\Delta H$ debe multiplicarse por el mismo factor.
                \ecuacion{2H_2O_{(s)} \rightarrow 2H_2O_{(l)} \hspace{15pt} \Delta H = 2 \times 6,01 \approx 12 kJ}
        \end{itemize}
        \begin{itemize}
            \item Los estados físicos de todos los reactivos y productos deben ser especificados en las ecuaciones termoquímicas.
                \ecuacion{H_2O_{\textcolor{red}{\textbf{(s)}}} \rightarrow H_2O_{\textcolor{red}{\textbf{(l)}}} \hspace{15pt} \Delta H = 6,01 kj/mol}
                \ecuacion{H_2O_{\textcolor{red}{\textbf{(l)}}} \rightarrow H_2O_{\textcolor{red}{\textbf{(g)}}} \hspace{15pt} \Delta H = 44,0 kj/mol}
        \end{itemize}
        \textbf{Conceptos:} \\
        \sangria{} El \textbf{calor específico (s)} de una sustancia es la cantidad de calor (q) requerido para elevar la temperatura de un gramo de sustancia en un grado Celsius. \\
        \sangria{} La \textbf{capacidad calorífica (C)} de una sustancia es la cantidad de calor (q) requerido para elevar la temperatura de una masa dada (m) de sustancia en un grado Celsius.
        \imagen{7cm}{./imagenes/capacidadCalorifica.png}
    \subsection{Entalpia estándar de formación}
    \sangria{} \textbf{Entalpia estándar de formación ($\Delta {H^o}_{f}$):} es el cambio de entalpía de la reacción que forma $1$ mol de compuesto a partir de sus elementos en condiciones de presión de $1atm$ y $25^o C$. \\
    \sangria{} \textbf{La entalpía estándar de formación de cualquier elemento en su forma más estable es $0$}.
        \ecuacion{ \Delta {H^o}_{f} (O_2) = 0 \hspace{15pt} \Delta {H^o}_{f} (C,grafito) = 0 }
        \ecuacion{ \Delta {H^o}_{f} (O_3) = 142 kJ/mol \hspace{15pt} \Delta {H^o}_{f}(C, diamante) = 1,90 kJ/mol }
    \sangria{} \textbf{Entalpia de combustión ($\Delta {H^o}_{c}$):} es el cambio de entalpía por mol de una sustancia que se quema en una reacción de combustión en condiciones estándar. \\
    Los productos de combustión de un compuesto orgánico son $CO_{2(g)}$ y $H_2O$, cualquier nitrógeno presente se libera como $N_2$ a menos que se especifiquen otros productos, \textcolor{red}{siempre es exotérmico}.
    \saltoPag{}
    \imagen{8cm}{./imagenes/tablaDeEntalpiaEstandar.png}
    \subsection{Entalpia estándar de reacción}
        \textcolor{blue}{\textbf{Entalpía estándar de reacción ($\Delta {H^o}$):}} se define como la cantidad de calor a presión constante de $1atm$ que el sistema puede absorber o liberar.
        \imagen{7cm}{./imagenes/entalpiaEstandarReaccion.png}
        \ecuacion{aA + bB + \dotsb \rightarrow cC + dD + \dotsb}
        \subsubsection{Método directo}
            \ecuacion{\Delta {H^o}_{r} = \sum n \Delta {H^o}_{prod} - \sum m \Delta {H^o}_{reac}}
            \ecuacion{\Delta {H^o}_{r} = [c\Delta H^o(C) + d\Delta H^o] - [a\Delta H^o(A) + b\Delta H^o(B)]}
            \sangria{} Para hacer el balance energético se considera que las reacciones tienen lugar hasta el consumo total de los reactivos. \\
            \textbf{\underline{Ejemplo:}} el benceno ($C_6H_6$) se quema en presencia de aire para producir el dióxido de carbono y el agua líquida ¿Cuánto calor se libera por cada mol de benceno quemado? La entalpía estándar de formación del benceno es $49,04 kJ / mol$.
            \imagen{7cm}{./imagenes/ejemploMetodoDirectoEntalpiaEstandarReaccion.png}
    \subsubsection{Método indirecto - Ley de Hess}
        \sangria{} La energía intercambiada en forma de calor en una reacción química es la misma tanto si la reacción ocurre en una etapa como si ocurre en varias.
        \ecuacion{C_{\text{(Grafito)}} + \frac{1}{2} O_{2 \text{(g)}} \rightarrow CO_{(g)}} 
        \begin{center}
            $C_{(Grafito)} + \cancel{O_{2(g)}}$ \hspace{25pt} $\textcolor{red}{\Delta {H^o}_{r} = - 393,5 kJ}$\\[5pt]
            $\cancel{CO_{2(g)}} \rightarrow CO_{(g)} + \frac{1}{2} \cancel{O_{2(g)}}$ \hspace{25pt} \textcolor{red}{$\Delta {H^o}_{r} = + 283 kJ$}\\[5pt]
            \rule{8cm}{0.05mm} \\[5pt]
            $C_{(Grafito)} + \frac{1}{2} O_{2(g)} \rightarrow CO_{(g)}$ \hspace{20pt} \textcolor{red}{$\Delta {H^o}_{r} = -110,5 kJ$}
        \end{center}
    \subsection{Función de estado}
        \begin{itemize}
            \item Tienen un valor único para cada estado del sistema.
            \item Su variación solo depende del estado final e inicial, no del camino desarrollado.
        \end{itemize}
        \textbf{\underline{SON:}} Presión, temperatura, volumen, energía interna, entalpía $\dotsb$ \\
        \textbf{\underline{NO SON:}} calor, trabajo $\dotsb$ \\[5pt]
        \textbf{\underline{Ejemplo:}} calcular la entalpía estándar de formación de $CS_{2(l)}$ a partir de las siguientes reacciones:
        \begin{center} 
            \textcolor{blue}{ $C_{(grafita)} + O_{2(g)} \rightarrow CO_{2(g)} \hspace{15pt} \Delta {H^o}_{rxn} = -393,5 kJ$ } \\[5pt]
            \textcolor{blue}{ $S_{\text{(\textit{rómbico})}} + O_{2(g)} \rightarrow SO_{2(g)} \hspace{15pt} \Delta {H^o}_{rxn} = -296,1 kJ $ } \\[5pt]
            \textcolor{blue}{ $CS_{2(l)} + 3O_{2(g)} \rightarrow CO_{2(g)} + 2SO_{2(g)} \hspace{10pt} \Delta {H^o}_{rxn} = -1072kJ$ }
        \end{center}
        \begin{enumerate}
            \item Escribir la reacción de formación para $CS_2$:
                \ecuacion{C_{(grafito)} + 2S_{(\text{\textit{rómbico}})} \rightarrow CS_{2(l)}}

            \item Sumar las tres ecuaciones algebraicamente.
            \begin{minipage}{8cm}
                \begin{center}
                \begin{enumerate}
                    \item $C_{(grafita)} + \cancel{O_{2(g)}} \rightarrow \cancel{CO_{2(g)}} $ \\[5pt]
                    \item $2S_{\text{(\textit{rómbico})}} + \cancel{2O_{2(g)}} \rightarrow \cancel{2SO_{2(g)}}$ \\[5pt]
                    \item $\cancel{CS_{2(l)}} + \cancel{3O_{2(g)}} \rightarrow CO_{2(g)} + \cancel{2SO_{2(g)}}$ \\[5pt]
                    +
                \end{enumerate}
                \rule{6cm}{0.05mm} \\[5pt]
                $C_{grafito} + 2S{\text{\textit{rómbico}}} \rightarrow CS_{2(l)}$
                \end{center}
            \end{minipage}
            \begin{enumerate}
                \item $\Delta {H^o}_{rxn} = -393,5 kJ$ 
                \item $\Delta {H^o}_{rxn} = -296,1 kJ \times 2$
                \item $\Delta {H^o}_{rxn} = -1072kJ \times (-1)$
                \item Respuesta: $\Delta {H^o}_{rxn} = -393,5 + 2 \times (-296,1)+ 1072 = \textcolor{red}{86,3 kJ}$
            \end{enumerate}
        \end{enumerate}
        \saltoPag{}

    \subsection{Entropía ($S$) y energía libre de Gibbs}
        \midTitle{red}{Procesos espontáneos}
        \sangria{} Los procesos que ocurren en forma espontánea en una dirección no pueden ocurrir en las mismas condiciones en dirección puesta. \\
        Para predecir la espontaneidad de un proceso se necesita conocer dos aspectos del sistema: $\Delta H$ y $S$.
        \imagen{7cm}{./imagenes/procesosEspontaneos.png}
        \midTitle{red}{\textbf{¿Una disminución en la entalpía significa que}} \vspace{-15pt}
        \midTitle{red}{\textbf{ una reacción sucede espontáneamente?}}
        \underline{Reacciones espontáneas} \\
        \imagen{7cm}{./imagenes/reaccionesEspontaneas.png}
        \sangria{} Es una medida del desorden o caos molecular del sistema que puede medirse y tabularse.
        \ecuacion{\Delta S = S_{final} - S_{inicial}}
        \sangria{} Existen tablas de $S^o$ (entropía molar estándar) de diferentes sustancias. \\
        \sangria{} La entropía es una función de estado.
        \ecuacion{S_{\text{\textit{Sólido}}} < S_{\text{\textit{Líquido}}} < S_{\text{\textit{gas}}}}
        \imagen{7cm}{./imagenes/procesosQueConducenAUnAumentoEntropia.png}

    \subsection{Segunda ley de la termodinámica}
        \sangria{} ''La entropía del universo aumenta en un proceso espontáneo y se mantiene constante en un proceso en equilibrio''.
        \begin{center} 
            \begin{tabular}{| m{3.1cm} | m{4.5cm} |}
                \hline
                Procesos espontáneos & $\Delta S_{universo} = \Delta S_{sistema} + \Delta S_{entorno} > 0$ \\
                \hline
                Procesos en equilibrio & $\Delta S_{universo} = \Delta S_{sistema} + \Delta S_{entorno} = 0$ \\
                \hline
            \end{tabular}
        \end{center}
        \sangria{} A veces el sistema pierde entropía (se ordena) espontáneamente. En dichos casos, el entorno se desordena.
        \midTitle{blue}{Cambios de entropía en el sistema}
        \ecuacion{aA + bB + \dotsb \rightarrow cD + dD + \dotsb}
        \ecuacion{\Delta {S^o}_{r} = [c\Delta {S^o}(C) + d\Delta S^o (D)] - [a\Delta S^o (A) + b\Delta S^o(B)]}
        \ecuacion{\Delta {S^o}_{r} = \sum n \Delta {S^o}_{prod} - \sum n \Delta {S^o}_{reac}}
        \subsubsection{Cambio de entropía en el sistema ($\Delta S_{sis}$)}
        \sangria{} Entropía estándar de reacción ($\Delta {S^o}_{rxn}$) es el cambio de entropía para una reacción realizada en $1atm$ y a $25^oC$.
        \ecuacion{aA + bB \rightarrow cC + dD}
        \begin{center}
            $\Delta {S^o}_{rxn} = [cS^o(C) + dS^o(D)] - [aS^o(A) + bS^o(B)]$ \\[5pt]
            $\Delta {S^o}_{rxn} = \sum n S^o(productos) - \sum m S^o(reactivos)$
        \end{center}
        \textbf{\underline{Ejemplo:}} ¿Cuál es el cambio de entropía estándar para la siguiente reacción a $25^oC$? 
        \ecuacion{2CO_{(g)} + O_{2(g)} \rightarrow 2CO_{2(g)}}
        \imagen{7cm}{./imagenes/ejemploCambioEntropiaSistema.png}
        
        \midTitle{blue}{Reglas generales}
        \ecuacion{CaCO_{3(s)} \rightarrow CaO_{(s)} + CO_{2(g)}}
        \begin{itemize}
            \item Si una reacción produce más moléculas gaseosas que las que consume, entonces $\Delta S^o > 0$.
                \ecuacion{N_{2(g)} + 3H_{2(g) \rightarrow 2NH_{3(g)}}}
            \item Si el número total de moléculas gaseosas disminuye, entonces $\Delta S^o < 0$.
                \ecuacion{H_{2(g)} + Cl_{2(g)} \rightarrow 2HCl_{(g)}}
            \item Si no hay ningún cambio neto del número total de moléculas gaseosas, entonces $\Delta S^o$ puede ser positivo o negativo, y tendrá un valor pequeño.
            \item Para sistemas donde solo hay líquidos y sólidos, la entropía aumenta con el número total de moléculas o iones.
        \end{itemize}
        \saltoPag{}
    \subsection{Tercera ley de la termodinámica}
        \sangria{} La entropía de una sustancia cristalina perfecta es de cero en el cero absoluto de temperatura ($0 K$). Los movimientos moleculares son mínimos, el número de micro-estados es 1 (solo hay $1$ forma de distribuir los átomos o moléculas para formar un cristal perfecto).
        \subsubsection{Cambio de entropía en el entorno $\Delta S_{(Entorno)}$}
            \imagen{7cm}{./imagenes/cambioEntropiaEntorno.png}
            \ecuacion{\Delta S_{Entorno} = \frac{-\Delta H_{Sistema}}{T_{Entorno}}}
            \ecuacion{\Delta S_{Univ} = \Delta S_{Sistema} + \Delta S_{Entorno}}
        \subsection{Energía libre de Gibbs}
            \ecuacion{\Delta S_{Univ} = \Delta S_{Sistema} + \Delta S_{Entorno} > 0}
            \ecuacion{\Delta S_{Univ} = \Delta S_{Sistema} - \frac{\Delta H_{Sistema}}{T} > 0}
            \ecuacion{-T\Delta S_{Univ} = \Delta H_{Sistema} - T\Delta S_{Sistema} < 0}
            \begin{center}
            \begin{tabular} {|m{5cm}|}
                \hline
                $\Delta G = \Delta H_{Sis} - T \Delta S_{sis}$ \\
                \hline
            \end{tabular}
            \end{center}
            \textbf{Valores de la energía libre de Gibbs:}
            \begin{itemize}
                \item $\Delta G < 0$: La reacción es espontánea.
                \item $\Delta G > 0$: La reacción no es espontánea, es inducida. 
                \item $\Delta G = 0$: La reacción está en equilibrio.
            \end{itemize}
            \subsubsection{Energía libre estándar de reacción ($\Delta {G^o}_{rxn}$)}
            \sangria{} La energía libre estándar de reacción ($\Delta {G^o}_{rxn}$) es el cambio de energía libre para una reacción cuando esto ocurre en condiciones estándar.
            \ecuacion{aA + bB \rightarrow cC + dD}
            \ecuacion{\Delta {G^o}_{rxn} = [c\Delta{G^o}_{f} (C) + d\Delta {G^o}_{f} (D)] - [a \Delta {G^o}_{f}(A) + b\Delta {G^o}_{f}(B)]}
            \ecuacion{\Delta {G^o}_{rxn} = \sum n \Delta {G^o}_{f}(productos) - \sum m \Delta {G^o}_{f}(reactivos)}
            \subsubsection{Energía libre estándar de formación ($\Delta G^o$)}
            \sangria{} La energía libre estándar de formación ($\Delta G^o$) es el cambio de energía libre que ocurre cuando $1mol$ del compuesto se forma a partir de sus elementos en estado estándar.
            La $\Delta G^o$ de cualquier elemento solo en su forma estable es cero.
    \subsection{Espontaneidad de las reacciones químicas}
        \ecuacion{\Delta G = \Delta H - T\Delta S}
        \sangria{} Según sean positivos o negativos los valores de $\Delta H$ y $\Delta S$ ($T$ siempre es positiva) se cumple que:
        \imagen{7cm}{./imagenes/espontaneidadReaccionesQuimicas.png}
        \rule{8cm}{0.05mm}
        \textbf{\underline{Ejemplo:}}
        \imagen{7cm}{./imagenes/ejemploEspontaneidadReaccionesQuimicas.png}
