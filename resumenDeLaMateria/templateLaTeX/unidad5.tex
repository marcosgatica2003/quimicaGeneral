\saltoPag{}
\section{UNIDAD 5}
    \subsection{Conceptos}
        \begin{itemize}
            \item \textbf{Energía:} capacidad para efectuar trabajo.
            \item \textbf{Energía térmica:} es la energía asociada con el movimiento arbitrario de átomos y moléculas.
            \item \textbf{Energía potencial:} es la energía disponible en virtud de la posición de un objeto.
            \item \textbf{Energía solar:} energía radiante proveniente del sol y es una fuente de energía primaria de la Tierra.
            \item \textbf{Energía química:} energía almacenada dentro de los enlaces de las sustancias químicas.
            \item \textbf{Termoquímica:} estudia los cambios de energía(calor) de las reacciones químicas.
        \end{itemize}
        \midTitle{blue}{Cambios de energía en las reacciones químicas}
        \begin{itemize}
            \item \textbf{Calor:} transferencia de energía térmica entre dos cuerpos que se encuentran a temperaturas diferentes.
            \item \textbf{Sistema:} parte pequeña del universo que se aísla para someter a estudio. Pueden ser:
                \begin{itemize}
                    \item Abiertos: intercambio de masa y energía.
                    \item Cerrados: intercambio de energía.
                    \item Aislados: no hay intercambio de masa ni de energía.
                \end{itemize}
                \imagen{7cm}{./imagenes/tiposDeSistemas.png}
            \item \textbf{Entorno o alrededores:} resto del universo externo al sistema.
        \end{itemize}
        \textbf{\underline{Ejemplo:}} la combustión del Hidrógeno gaseoso con oxígeno, libera gran cantidad de energía.
        \begin{center} 
            $2H_{2(g)} + O_{2(g)} \rightarrow 2H_2O_{l} +$ ENERGÍA. 
        \end{center}
        \sangria{} La mezcla de reacción es un sistema y el resto del universo, los alrededores. El calor generado por el sistema lo toma los alrededores. El proceso es \textbf{Exotérmico}.
        \sangria{} La descomposición de óxido de Mercurio, es un proceso endotérmico.
        \imagen{7cm}{./imagenes/combustionHidrogenoOxigenoYDescomposicionOxidoMercurio.png}
        \begin{itemize}
            \item \textbf{Trabajo:} cambio de energía producido por un proceso.
            \begin{itemize}
                \item Signo del Trabajo:
                    \begin{itemize}
                        \item Compresión ($\Delta V < 0$), signo ($-$) $\Rightarrow W (+)$
                        \item Expansión ($\Delta V > 0$), signo ($+$) $\Rightarrow W(-)$
                    \end{itemize}
            \end{itemize}
        \end{itemize}
        \imagen{7cm}{./imagenes/deltaVolumenTrabajoGas.png}
        \midTitle{blue}{Calor a volumen constante ($q_v$)}
        \begin{center} $\Delta E = q + W$ \\[5pt] \end{center}
        Siendo:
        \begin{center} $W = - P \Delta V$ \\[5pt] $W = 0$ \end{center}
        Entonces:
        \begin{center}
            \begin{tabular}{| c |}
                \hline 
                $\Delta E = q_V$ \\ 
                \hline 
            \end{tabular} 
        \end{center}
        \midTitle{blue}{Calor a presión constante ($q_P$): Entalpia}
        \begin{center} $\Delta E = q + W$ \end{center}
        Siendo:
        \begin{center}
            $q_P = \Delta H$ \\[5pt]
            $W = -P \Delta V$
        \end{center}
        Entonces:
        \begin{center}
            $\Delta E = \Delta H - P \Delta V$ \\[5pt]
            \begin{tabular}{| c |}
                \hline
                $\Delta H = \Delta E + P \Delta V $
            \end{tabular}
        \end{center}
        Para sistemas gaseosos:
        \begin{center} \begin{tabular}{| c |} \hline $\Delta H = \Delta E + R \cdot T \cdot \Delta n$ \end{tabular} \end{center}
