\saltoPag{}
\section{UNIDAD 9}
    \subsection{Electroquímica}
        \sangria{} Las reacciones de oxidación - reducción o reacciones redox, se consideran como reacciones de transferencia de electrones.
        \imagen{6cm}{./imagenes/redox.png}
        Los procesos electroquímicos consisten en reacciones de oxido-reducción con las cuales: \begin{itemize} \item La energía liberada por una reacción espontánea es convertida en electricidad. \item La energía eléctrica es usada para hacer que una reacción no espontánea ocurra. \end{itemize}
    \subsection{Número de oxidación}
    \sangria{} Es la carga que un átomo tendría en una molécula (o en un compuesto iónico) si los electrones fueran transferidos completamente.
    \begin{enumerate}
        \item Los elementos libre (sin combinar) en su estado más estable tienen un número de oxidación igual a cero.
            \ecuacion{Na, Be. K. Pb. H_2, O_2, P_4 = 0}
        \item En iones monoatómicos, el número de oxidación es igual a la carga del ión.
            \ecuacion{Li^+, Li = +1; Fe^{3+}, Fe = +3; O^{2-}, O = -2}
        \item El número de oxidación del oxígeno es por lo general $-2$, excepto en peróxidos (ej. $H_2O_2$) donde es -1.
        \item El número de oxidación del hidrógeno es $+1$ excepto cuando esto es vinculado a metales en compuestos binarios. En estos casos, su número de oxidación es $-1$.
        \item Los metales del grupo $IA$ tienen $+1$, los metales del $IIA$ tienen $+2$ y el del flúor es siempre $-1$.
        \item La suma de los números de oxidación de todos los átomos en una molécula o en un ión es igual a la carga de la molécula o del ión.
    \end{enumerate}
    \subsubsection{Tipos de reacciones de oxidación-reducción}
        \midTitle{red}{Reacción de formación}
        \ecuacion{A + B \rightarrow C}
        \ecuacion{2\overset{0}{Al} + 3\overset{0}{Br_2} \rightarrow 2\overset{+3}{Al}\overset{-1}{Br_3}}
        \midTitle{blue}{Reacción de descomposición}
        \ecuacion{C \rightarrow A + B}
        \ecuacion{2\overset{+1}{K}\overset{+5}{Cl}\overset{-2}{O_3} \rightarrow 2\overset{+1}{K}\overset{-1}{Cl} + 3\overset{0}{O_2}}
        \midTitle{purple}{Reacción de combustión}
        \ecuacion{A + O_2 \rightarrow B}
        \ecuacion{\overset{0}{S} + \overset{0}{O_2} \rightarrow \overset{+4}{S}\overset{-2}{O_2}}
        \ecuacion{2\overset{0}{Mg} + \overset{0}{O_2} \rightarrow 2\overset{+2}{Mg}\overset{-2}{O}}
        \midTitle{teal}{Reacciones de desplazamiento}
        \ecuacion{A + BC \rightarrow AC + B}
        \textbf{Desplazamiento del hidrógeno:}
        \ecuacion{\overset{0}{Sr} + 2\overset{+1}{H_2}O \rightarrow \overset{+2}{Sr}(OH)_2 + \overset{0}{H_2}}
        \textbf{Desplazamiento del metal:}
        \ecuacion{\overset{+4}{Ti}Cl_4 + 2\overset{0}{Mg} \rightarrow \overset{0}{Ti} + 2\overset{+2}{Mg}{Cl_2}}
        \textbf{Desplazamiento del halógeno:}
        \ecuacion{\overset{0}{Cl_2} + 2K\overset{-1}{Br} \rightarrow 2K\overset{-1}{Cl} + \overset{0}{Br_2}}
        Series de actividad para los halógenos:
        \ecuacion{F_2 > Cl_2 > Br_2 > I_2}
        Reacción de desplazamiento del halógeno:
        \ecuacion{\overset{0}{Cl_2} + 2K\overset{-1}{Br} \rightarrow 2K\overset{-1}{Cl} + \overset{0}{Br_2}}
        \ecuacion{\cancel{I_2 + 2KBr \rightarrow 2KI + Br_2}}
        \subsubsection{Reglas de disociación}
        \sangria{} Los compuestos que se disocian en una reacción de óxido reducción son los ácidos, sales e hidróxidos. \\
        \sangria{} Los óxidos, peróxidos y sustancias simples (ej. $O_2$) no se disocian. \\[10pt]
        \textbf{OXIDACIÓN:} pérdida de electrones (aumento en el número de oxidación).
        \ecuacion{Cu \rightarrow Cu^{+2} + 2e^-}
        \textbf{REDUCCIÓN:} ganancia de electrones (disminución del número de oxidación).
        \ecuacion{Ag^+ + e^- \rightarrow Ag}
        \sangria{} Siempre que se produce una oxidación debe producirse simultáneamente una reducción. Cada una de estas reacciones se denomina ''semireacción''.
        \imagen{6cm}{./imagenes/numerosDeOxidacion.png} \saltoPag{}
        \textbf{AGENTE OXIDANTE:} sustancia capaz de oxidar a otro, por lo tanto ésta se reduce. \\
        \textbf{AGENTE REDUCTOR:} sustancia capaz de reducir a otra, por lo tanto esta se oxida.
        \ecuacion{\text{Ej.\hspace{5pt}} Pb(NO_3)_2 + Zn \rightarrow Zn(NO_3)_{2(ac)} + Pb_{(s)}}
        \begin{center}
            \begin{tabular}{cc}
                \multicolumn{1}{c}{$Zn \rightarrow Zn^{2+} + 2e^-$} &
                \multicolumn{1}{c}{\textcolor{red}{Semirreacción de oxidación.}} \\
                \multicolumn{1}{c}{$Pb^{2+} + 2e^- \rightarrow Pb$} &
                \multicolumn{1}{c}{\textcolor{red}{Semirreacción de reducción.}} \\
            \end{tabular}
        \end{center}
        Agente reductor: $Zn$ (se oxida de $0$ a $+2$.) \\
        Agente oxidante: $Pb(NO_3)_2$ (el $Pb$ se reduce de $+2$ a $0$.)
    \subsection{Balances redox en medio ácido}
    \textbf{\underline{Ejemplo:}}
    \ecuacion{K\overset{+7}{Mn}O_4 + H_2SO_4 + H_2\overset{-1}{O_2} \rightarrow \overset{+2}{Mn}SO_4 + \overset{0}{O_2} + H_2O `K_2SO_4}
    \begin{enumerate}
        \item Identificar los elementos que cambian su estado de oxidación.
        \item Plantear dos semireacciones con las especies que cambian su estado de oxidación.
            \begin{center}
                \begin{tabular}{cc}
                    \multicolumn{1}{c}{\textcolor{red}{Oxidación:}} &
                    \multicolumn{1}{c}{$H_2\overset{-1}{O_2} \rightarrow \overset{0}{O_2}$} \\[5pt]
                    \multicolumn{1}{c}{\textcolor{blue}{Reducción:}} &
                    \multicolumn{1}{c}{$\overset{+7}{Mn}O_4 \rightarrow \overset{+2}{Mn}$} \\
                \end{tabular}
            \end{center}
        \item Balancear por inspección todos los elementos que no sean ni oxígeno ni hidrógeno en las dos semireacciones.
        \item Para reacciones en medio ácido, agregar $H_2O$ para balancear los átomos de $O$. Para balancear los átomos de $H$, agregamos $H^+.$ \\
        Semirreacción de reducción:
        \ecuacion{(8H^{1+} + MnO^-1_4 \rightarrow Mn^{2+} + 4H_2O)}
        Semirreacción de oxidación:
        \ecuacion{H_2O_2 \rightarrow O_2 + 2H^{1+}}
    \item Agregar electrones en el lado apropiado de cada una de las semireacciones para balancear las cargas: \\[5pt]
        Reducción:
        \ecuacion{5e^- + 8H^{1+} + MnO^{-1}_4 \rightarrow Mn^{2+} + 4H_2O}
        Oxidación:
        \ecuacion{H_2O_2 \rightarrow O_2 + 2H^{1+} + 2e^-}
    \item Si es necesario, igualar el número de electrones en las dos semireacciones multiplicando cada una de las reacciones por un coeficiente apropiado: \\[5pt]
        Reducción:
        \ecuacion{(5e^- + 8H^{1+} + MnO^{-1}_4 \rightarrow Mn^{2+} + 4H_2O) \times 2}
        \columnbreak{}
        Oxidación:
        \ecuacion{(H_2O_2 \rightarrow O_2 + 2H^{1+} + 2e^-) \times 5}

    \item Se cancelan los electrones en ambas partes. Se suman los reactivos y productos de ambas hemireacciones.
    \item Verificar que el número de átomos y las cargas estén balanceadas.
    \end{enumerate}
    \imagen{8cm}{./imagenes/finalBalanceRedoxMedioAcido.png}

    \subsection{Celdas galvánicas}
        \imagen{7cm}{./imagenes/celdasGalvanicas.png}
        \imagen{5cm}{./imagenes/dibujoCeldaGalvanica.png}
        \sangria{} Las reacciones de reducción siempre tienen lugar en el cátodo. \\ \sangria{} Las reacciones de oxidación siempre tienen lugar en el ánodo.
        \begin{center}
            \begin{tabular}{|c|}
                \toprule
                \multicolumn{1}{c}{Por convención} \\ \midrule
                \multicolumn{1}{c}{El cátodo corresponde al polo positivo de la pila.} \\
                \multicolumn{1}{c}{El ánodo corresponde al polo negativo de la pila.} \\ \bottomrule
            \end{tabular}
        \end{center}
        \sangria{} El punte salino se utiliza para unir los dos compartimientos de los electrodos y completar el circuito eléctrico. El más utilizado es el $KCl$.
        \subsubsection{Celdas electroquímicas, galvánicas o voltaicas}
        \sangria{} La diferencia de potencial eléctrico entre el ánodo y el cátodo se llama: \saltoPag{}
        \begin{itemize}
            \item Voltaje de la celda.
            \item Fuerza electromotriz ($fem$) o
            \item Potencial de la celda
        \end{itemize}
        \midTitle{red}{Diagramas de celdas}
        \ecuacion{Zn_{(s)} + Cu^{2+}_{(ac)} \rightarrow Cu_{(s)} + Zn^{2+}_{(ac)}}
        \ecuacion{[Cu^{2+}] = 1M \hspace{15pt} [Zn^{2+}] = 1M}
        \ecuacion{Zn_{(s)} | Zn^{2+} (1M) || Cu^{2+} (1M) | Cu_{(s)}}
        \begin{center} \hspace{5pt}\textcolor{red}{Ánodo} \hspace{30pt}\textcolor{blue}{Cátodo} \end{center}

        \subsubsection{Potenciales estándar de reducción}
        \sangria{} El potencial estándar de reducción ($E^0$) es el voltaje asociado con una reacción de reducción en un electrodo cuando la concentración de iones es $1M$ y la presión de los gases es de $1atm$. \\
        \sangria{} El electrodo de $Pt$ proporciona la superficie para la disociación de $H_2$ y la transferencia externa de electrones.
        \imagen{4cm}{./imagenes/potencialEstandarDeReduccion.png}
        \begin{center} \textit{Electrodo estándar de hidrógeno ($EEH$)} \end{center}
        \ecuacion{2e^- + 2H^+ (1M) \rightarrow H_2 (1atm)}
        \ecuacion{E^0 = 0V}
        \sangria{} Arbitrariamente se asigna potencial $E = 0V$ para poder determinar los potenciales relativos de otros electrodos.
        \imagen{8cm}{./imagenes/potencialEstandarDeReduccionDelZn.png}
        \imagen{8cm}{./imagenes/tablaDePotencialesEstandar.png}
        \begin{itemize}
            \item El valor de $E^o$ para cada semireacción de reducción aparece en la tabla de potenciales de reducción.
            \item Cuando $E^o$ sea más positivo, mayor será la tendencia de la sustancia para ser reducida.
            \item Las semireacciones son reversibles.
            \item El signo de $E^o$ se cambia cuando la reacción se invierte.
            \item La variación de los coeficientes estequiométricos de una semireacción no altera el valor de $E^o$.
        \end{itemize}
        \sangria{}\textcolor{blue}{¿Cuál es la fem estándar de una celda electroquímica hecha de un electrodo $Cd$ en una solución de $Cd(NO_3)_2$ y un electrodo $Cr$ en una solución de $Cr(NO_3)_3$?}
        \imagen{8cm}{./imagenes/femEstandarElectrodoCd.png}
        \saltoPag{}
        \subsubsection{Espontaneidad de reacciones redox}
        \sangria{} En una celda galvánica, la energía química se transforma en energía eléctrica. La energía es el producto de la fem de la celda por la carga eléctrica total (en Coulombs) que pasa a través de la celda.
        \ecuacion{\text{Energía eléctrica } = [Volt] \times [Coulombs] = [Joules]}
        \ecuacion{1J = 1C \times 1V}
        \ecuacion{\text{Carga total } = n^o e^- \times \text{Carga de un } e^-}
        $F$ = constante de Faraday = Carga de $1mol$ de electrones.
        \ecuacion{1F = 96599 \frac{C}{mol} \hspace{15pt} 1F = 96500 \frac{J}{V\cdot mol}}
        \ecuacion{W_{\text{Eléctrico}} = -nFE_{\text{celda}}}
        \ecuacion{\Delta G = W_{\text{Eléctrico}}}
        \ecuacion{\Delta G = -nFE_{\text{celda}}}
        \imagen{5cm}{./imagenes/formulasParaDeltaGKECelda.png}
        \ecuacion{\Delta G^0 = -RT\ln(K) = -nFE^0_{\text{celda}} \hspace{15pt} \text{En condiciones estándar.}}
        \ecuacion{E^0_{cel} = \frac{RT}{nF}\ln(k) = \frac{(8,314J/K \cdot mol)(298K)}{n(96500 J/V \cdot mol)} \ln(K)}
        \ecuacion{E^0_{cel} = \frac{0,0257V}{n}\ln(K)}
        \ecuacion{E^0_{cel} = \frac{0,0592V}{n}\log(K)}
        \imagen{5cm}{./imagenes/tablaEspontaneidadReaccionesRedox.png}
        \begin{center} $\Delta G^0 = -RT\ln(K) = -nFE^o_{celda}$ \end{center}
        \textbf{\underline{Ejemplo:}} \\[5pt]
        \textcolor{blue}{¿Cuál es la constante de equilibrio para la reacción siguiente a $25^oC$?}
        \ecuacion{Fe^{2+}_{(ac)} + 2Ag_{(s)} \rightleftharpoons Fe_{(s)} + 2Ag^+_{(ac)}}
        \ecuacion{E^0_cel = \frac{0,0257V}{n} \ln(K)}
        \imagen{8cm}{./imagenes/ejemploCalculoConstanteDeEquilibrio.png}
        \midTitle{red}{Efecto de la concentración sobre la FEM de la celda}
        \ecuacion{\Delta G = \Delta G^0 + RT \ln(Q)}
        \ecuacion{\Delta G = -nFE \hspace{15pt} \Delta G^0 = -nFE^0}
        \ecuacion{-nFE = -nFE^0 + RT\ln(Q)}
        \begin{center}
            \begin{tabular}{|c|}
                \toprule
                \multicolumn{1}{|c|}{\textbf{Ecuación de Nernst}} \\ \midrule
                \multicolumn{1}{|c|}{$E = E^0  -\frac{RT}{nF}\ln(Q)$} \\ \bottomrule
            \end{tabular}
        \end{center}
        A $298K$:
        \ecuacion{E = E^0 - \frac{0,0257V}{n}\ln(Q) = E^0 - \frac{0,0592V}{n}\log(Q)}
        \textbf{\underline{Ejemplo:}} \\[5pt]
        \sangria{}\textcolor{blue}{¿Tendrá lugar la siguiente reacción de forma espontánea a $25^0C$?}
        \ecuacion{Fe^{2+}_{(ac)}(0,6M) + Cd_{(s)} \rightarrow Fe_{(s)} + Cd^{2+}_{(ac)}(0,01M)}
        \imagen{8cm}{./imagenes/ejemploReaccionDeFormaEspontanea.png}
        \textbf{\underline{Ejemplo:}} \\[5pt]
        Para la siguiente pila:
        \ecuacion{Al/Al^{3+}(0,1M) // Ni^{2+}(0,01M)/Ni}
        \begin{enumerate}
            \item Calcular la FEM de la celda.
            \item Indicar si la misma se produce espontáneamente.
            \item ¿En qué electrodo se produce el depósito?
        \end{enumerate}
        \imagen{8cm}{./imagenes/ejemploReaccionDeFormaEspontaneaEnUnaPila.png} \saltoPag{}
        \subsubsection{Baterías}
        \textbf{Baterías y pilas:} dispositivos que almacenan energía química para ser liberada como electricidad. Se considerará tres tipos de pilas y baterías: \\[5pt]
        \underline{Baterías primarias o pilas (celdas primarias):} la reacción de la celda no es reversible. \\[5pt]
        \underline{Baterías secundarias (celdas secundarias):} la reacción de la celda puede revertirse haciendo pasar electricidad a través de la batería. \\[5pt]
        \underline{Baterías de flujo y celdas de combustible:} los materiales pasan a través de la batería. Pueden usarse indefinidamente.
        \imagen{8cm}{./imagenes/celdaDeLeclanche.png}
        \midTitle{red}{Batería de mercurio}
        \sangria{} Se utiliza en medicina e industria electrónica. Es más costosa que la pila seca. Contiene un ánodo de $Zn$ (amalgamado con $Hg$). El voltaje es más constante ($1,35V$) y posee mayor vida útil. Esto la hace ideal para marcapasos, aparatos auditivos, relojes, fotómetros.
        \imagen{6cm}{./imagenes/bateriaDeMercurio.png}
        \midTitle{blue}{Acumuladores de plomo}
        \sangria{} Consiste en $6$ celdas en serie con ánodo de $Pb$ y cátodo de $PbO_2$. El electrolito es $H_2SO_4$. Cada celda produce $2V$. La batería es re-cargable revirtiendo las reacciones mediante la aplicación de voltaje externo. \\ \sangria{} Midiendo la densidad del ácido se sabe cuánto se ha descargado la batería:
        \begin{itemize}
            \item $1,2 g/ml$: es la densidad normal, a baja temperatura la densidad y viscosidad aumenta, disminuyendo la movilidad de carga y la energía suministrada.
        \end{itemize}
        \imagen{4cm}{./imagenes/acumuladorDePlomo.png}
        \imagen{8cm}{./imagenes/ecuacionAcumuladorDePlomo.png}
        \midTitle{teal}{Batería de ión de litio}
        \sangria{} Debido a la alta reactividad del metal se usa un electrolito no acuoso como disolvente orgánico y sal disuelta. La ventaja es que el $LI$ tiene el potencial estándar de reducción más negativo y por lo tanto la mayor fuerza de agente reductor, también cuenta con un bajo $PM (g/mol)$. Usos en teléfonos celulares, computadoras, cámaras.
        \imagen{7cm}{./imagenes/bateriaDeIonDeLitio.png}
        \midTitle{blue}{Celda de combustible}
        \sangria{} Es una celda electroquímica que requiere un suministro continuo de reactivos para seguir funcionando y eliminar los productos. Una celda de oxígeno e hidrógeno está compuesta por dos electrodos inertes y una disolución electrolítica ($KOH$). En estado estándar la fem de la celda es $1,23V$.
        \imagen{9cm}{./imagenes/celdaDeCombustible.png}
        \subsubsection{Electrólisis}
            \sangria{} Es el proceso en el que se usa energía eléctrica para hacer que una reacción química no espontánea tenga lugar. La polaridad del ánodo es ($+$) y la del cátodo es ($-$).
            \saltoPag{}
            \imagen{8cm}{./imagenes/electrolisisDeNaClFundido.png}
            \imagen{8cm}{./imagenes/electrolisisDelAgua.png}
            \imagen{8cm}{./imagenes/electrolisisDeNaClAcuoso.png}
        \subsubsection{Electrólisis y cantidad de masa}
        \textit{Carga (Coulomb) = Intensidad de corriente (Amper) $\times$ Tiempo (segundos)}
        \ecuacion{Carga = i \times t \hspace{15pt} [C] = [A] \times [s]}
        \recuadrar{5cm}{Carga de $1mol e^- = 96500 C = 1F$}
        \imagen{8cm}{./imagenes/electrolisisYCantidadDeMasa.png}
        \textbf{\underline{Ejemplo:}} \\
        \textcolor{blue}{¿Cuánto calcio se producirá en una celda electrolítica de $CaCl_2$ fundido si se aplica una corriente de $0,452A$ durante $1,5$ horas?}
        \imagen{8cm}{./imagenes/ejercicioElectrolisisYCantidadDeMasa.png}
    \subsection{Corrosión}
    \sangria{} Se refiere al deterioro de los metales por un proceso electroquímico.
    \imagen{8cm}{./imagenes/corrosion.png}
    Redox global:
    \ecuacion{2Fe_{(s)} + O_{2(g)} + 4H^+_{(ac)} \rightarrow 2Fe^{2+}_{(ac)} + H_2O_{(l)}}
    Los iones $Fe^{2+}$ formados en el ánodo se oxidan con el oxígeno:
    \ecuacion{4Fe^{2+}_{(ac)} + O_{2(g)} + (4 + 2x)H_2O_{(l)} \rightarrow 2Fe_2O_3 \cdot xH_2O_{(s)} + 8H^+_{(ac)}}
    \textbf{Patinas de $CuCO_3:$} reacción del $Cu$ con el $CO_2$ forma carbonato protector de la superficie. \\
    \textbf{Latón:} aleación de $Cu-Zn$ \\
    \textbf{Bronce:} aleación de $Cu-Sn$
    \imagen{5cm}{./imagenes/latonYBronce.png}
    \textbf{Herrumbre:} oxidación del $Fe$. \\
    \textbf{Formación de $Ag_2S$:} (protector) por contacto con alimentos (cubiertos). \\
    El $Al$ forma una delgada capa superficial de $Al_2O_3$. Este óxido es de alta dureza e inerte al oxígeno, agua y casi todos los demás agentes corrosivos del ambiente.
    \subsubsection{Métodos de protección de metales}
    \sangria{} La tendencia del $Fe$ a oxidarse disminuye al alearse con otros metales. El acero inoxidable, es una aleación de $Fe-Cr$, la capa de óxido de $Cr$ que se forma, protege al $Fe$ de la corrosión.
    \saltoPag{}
    \begin{itemize}
        \item \textbf{Pintura:} se cubre la superficie con una pintura epoxi para aislar el metal de las condiciones oxidantes del ambiente (humedad y oxígeno). La desventaja cuando se rasga o descargara la superficie y se deja expuesto el metal, este se oxida.
        \imagen{6cm}{./imagenes/proteccionPintura.png}
    \item \textbf{Inactivación o pasivación:} tratamiento de la superficie por medio de un oxidante fuerte para la generación de óxidos protectores.
        \imagen{7cm}{./imagenes/proteccionPasivacion.png}
    \item \textbf{Galvanizado:} un ejemplo es cubrir el $Fe$ con $Sn$ o $Zn$.
        \imagen{7cm}{./imagenes/proteccionGalvanizado.png}
        \sangria{} Cuando la capa de estaño o cobre se rasga, el hierro que está abajo se corroe aún con más rapidez que si estuviera descubierto, debido a la celda electroquímica adversa que se genera. La oxidación preferencial del $Zn$ antes que el $Fe$ genera una capa de óxido protector sobre la superficie:
        \begin{center}
        \begin{tabular}{cc}
        \toprule
        \multicolumn{1}{c}{$Cu^{2+}_{(ac)} + 2e^- \rightarrow Cu_{(s)}$} &
        \multicolumn{1}{c}{$E^0 = +0,33V$} \\[5pt]
        \multicolumn{1}{c}{$Sn^{2+}_{(ac)} + 2e^- \rightarrow Sn_{(s)}$} &
        \multicolumn{1}{c}{$E^0 = -0,14V$} \\[5pt]
        \multicolumn{1}{c}{$Fe^{2+}_{(ac)} + 2e^- \rightarrow Fe_{(s)}$} &
        \multicolumn{1}{c}{$E^0 = -0,44V$} \\[5pt]
        \multicolumn{1}{c}{$Zn^{2+}_{(ac)} + 2e^- \rightarrow Zn_{(s)}$} &
        \multicolumn{1}{c}{$E^0 =-0,76V$} \\[5pt] \bottomrule
        \end{tabular}
        \end{center}
        Nótese que en la segunda ecuación, el $Sn$ se reduce frente al $Fe$. Mientras que e la última ecuación, el $Zn$ se oxida frente al $Fe$ y genera óxidos protectores.
    \item Protección catodica de un tanque de hierro:
        \imagen{7cm}{./imagenes/proteccionCatodicaDeUnTanqueDeHierro.png}
    \end{itemize}

